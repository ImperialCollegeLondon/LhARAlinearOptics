\section*{Introduction}
This document summarises the steps needed to set-up and run
the linear optics simulation of the LhARA beamine.
A summary of the tasks that LhARA\_Beamline software suite performs
will be documented in due course.
The code has been developed in python; python~3 is assumed.

\section*{Getting the code}
The linear optics package is maintained using the GitHub
version-control system.
The latest release can be downloaded from: \\
\centerline{
  \href{https://github.com/ImperialCollegeLondon/LhARAlinearOptics.git}https://github.com/ImperialCollegeLondon/LhARAlinearOptics.git{}
}

\section*{Dependencies and required packages}
The package requires the following packages:
\begin{itemize}
  \item Python modules: \verb+scipy+ and\verb+matplotlib+. 
\end{itemize}
It may be convenient to run LhARA\_Beamline in a ``virtual environment''.
To set this up, after updating your python installation to python~3.9,
execute the following commands:
\begin{enumerate}
  \item \verb+python3 -m venv --system-site-packages venv+
    \begin{itemize}
      \item This creates the director \verb+venv+ that contains files
        related to the virtual environment.
    \end{itemize}
  \item \verb+source venv/bin/activate+
  \item \verb+python -m pip install pandas scipy matplotlib+
\end{enumerate}
To exit from the virtual environment, execute the command
\verb+deactivate+. \\
\noindent
The command \verb+source venv/bin/activate+ places you back
into the virtual environment.

Note that the Imperial HEP linux cluster provides python 3.9.18 by
default.

\section*{Unpacking the code, directories, and running the tests}
After downloading the package from GitHub, or cloning the repository,
you will find a ``\verb+README.md+'' file which provides some orientation
and instructions to run the code.
In particular, a \verb+bash+ script ``\verb+startup.bash+'' is
provided which:
\begin{itemize}
  \item Sets the ``\verb+LhARAOpticsPATH+'' environment variable
    so that the files that hold constants etc. required by the code
    can be located; and
  \item Adds ``\verb+01-Code+'' (see below) to the PYTHONPATH.
    The scripts in "02-Tests" (see below) may then be run with the
    command "python 02-Tests/\textless\,filename\,\textgreater.py".
\end{itemize}
Below the top directory, the directory structure in which the code is
presented is:
\begin{description}
  \item\verb+01-Code+: contains the python implementation as a
    series of modules.
    Each module contains a single class or a related set of methods.
  \item\verb+02-Tests+: contains self-contained test scripts that
    run the various methods and simulation packages defined in the
    code directory.
  \item\verb+11-Parameters+: contains the parameter set used in
    \verb+02-Tests/RunSimulation.py+ to generate muon decays in
    the production straight.
\end{description}
The instruction in the \verb+README.md+ file should be followed to set
up and run the code.

\section*{Running the code}
Execute "startup.bash" from the top directory (i.e. run the bash
command "source startup.bash").  This will:
\begin{itemize}
  \item Set up "LhARAOpticsPATH"; and
  \item Add "01-Code" to the PYTHONPATH.  The scripts in "02-Tests"
    may then be run with the command "python 02-Tests/<filename>.py";
  \item Example scripts are provided in "03-Scripts'', these can be
    used first to ``Run'' the simulation and then to ``Read'' the data
    file produced.
    Example scripts are provided for the DRACO, LION, and LhARA Stage
    1 beam lines.
\end{itemize}
