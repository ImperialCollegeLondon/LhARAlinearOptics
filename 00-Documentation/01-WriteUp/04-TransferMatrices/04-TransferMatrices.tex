\graphicspath{ {04-TransferMatrices/Figures/} }

\section{Transfer matrices}

Description of beam transport often carried out by:
\begin{itemize}
  \item Breaking lattice down into a series of ``elements'',
    e.g. drift, quadrupole, dipole, solenoid, etc.;
  \item Transport of particle between the start and end of a
    particular element of the lattice linearised such that the trace
    space at the end of the element, $\bm{\phi}_f$, is written in
    terms of the trace space at the start of the element,
    $\bm{\phi}_i$:
    \begin{equation}
      \bm{\phi}_f = \underline{\underline{T}} ~ \bm{\phi}_i \,
      {\rm ; and}
    \end{equation}
  \item The distance along the reference particle trajectory is
    increased by $\delta s$, where:
    \begin{equation}
      s_{\rm end} = s_{\rm start} + \delta s \, ;
    \end{equation}
    and $s_{\rm start}$ and $s_{\rm end}$ are the length of the
    reference particle trajectory at the start and end of the 
    beam-line element respectively.
\end{itemize}
Many excellent descriptions of derivation of transfer matrices,
$\underline{\underline{T}}$, so only quote results here.

\subsection{Drift}

A ``drift'' space refers to a region in which the beam propagates in
the absence of any electromagnetic fields.
In a drift, particles propagate in straight lines, therefore:
\begin{equation}
  \underline{\underline{T}}_{\rm~drift} =
        \begin{pmatrix}
          1 & l & 0 & 0 & 0 &                             0 \\
          0 & 1 & 0 & 0 & 0 &                             0 \\
          0 & 0 & 1 & l & 0 &                             0 \\
          0 & 0 & 0 & 1 & 0 &                             0 \\
          0 & 0 & 0 & 0 & 1 & \frac{l}{\beta_0^2 \gamma_0^2} \\
          0 & 0 & 0 & 0 & 0 &                             1
        \end{pmatrix} \, ; 
\end{equation}
where $l$ is the length of the drift.
The increment in the reference particle trajectory is:
\begin{equation}
  \delta s = l \, .
\end{equation}

\subsection{Quadrupole}

The passage of a beam particle through a quadrupole magnet may be
described by specifying the field gradient, $g$, within the magnet and
the length, $l_q$, of the quadrupole measured along its axis.
The impact of a quadrupole on the trajectory of a particle in the $xy$
plane is independent of the impact of the magnet on the particle's
trajectory in the $yz$ plane.   
In this sense quadrupole focusing in the $xz$ and $yz$ planes is said
to be ``uncoupled''. 

If the field gradient along the $x$ and $y$ axes is identical, then:
\begin{equation}
  g_x = \frac{\partial B_{qx}}{\partial x} =
  g_y = \frac{\partial B_{qy}}{\partial y} = g \, ; 
\end{equation}
where the field in the quadrupole, $\bm{B_q}$, has components
$(B_{qx}, B_{qy}, 0)$.

In the ``hard-edge'' approximation, where the field falls to zero at
the start and end of the quadrupole, the transfer matrix for a
quadrupole focusing in the $xz$ plane (a ``focusing quadrupole'') may
be written: 
\begin{equation}
  \underline{\underline{T}}_{\rm~Fquad} =
    \begin{pmatrix}
          \cos(\sqrt{k_q} l_q) & \frac{\sin(\sqrt{k_q} l_q)}{\sqrt{k_q}} & 0 & 0             & 0 & 0 \\
-\sqrt{k_q} \sin(\sqrt{k_q} l_q) &                  \cos(\sqrt{k_q} l_q) & 0 & 0             & 0 & 0 \\
          0 & 0 &           \cosh(\sqrt{k_q} l_q) & \frac{\sinh(\sqrt{k_q} l_q)}{\sqrt{k_q}} & 0 & 0 \\
          0 & 0 &  \sqrt{k_q} \sinh(\sqrt{k_q} l_q) &                  \cosh(\sqrt{k_q} l_q) & 0 & 0 \\
          0 & 0 & 0 & 0 & 1 & \frac{l_q}{\beta_0^2 \gamma_0^2} \\
          0 & 0 & 0 & 0 & 0 &                             1
        \end{pmatrix} \, ; 
\end{equation}
where:
\begin{equation}
  k_q = \frac{gc}{p} \times 10^{-3} \, ,  \label{Eq:Effectivekq}
\end{equation}
and $c$ is the speed of light in metres per second, p is the magnitude
of the momentum of the particle in MeV/c, and the field gradient, $g$,
is given in T/m.
As before, $\beta_0$ is the relativistic velocity of the reference
particle and $\gamma_0=(1-\beta_0^2)^{-\frac{1}{2}}$.
The increment in the reference particle trajectory is:
\begin{equation}
  \delta s = l_q \, .
\end{equation}

It is important to include a description of the effect of dispersion
on focusing in the LhARA beam line since the laser-driven proton and
ion source provides a broad energy spectrum.
Reference \cite{Wolski:2014} provides two methods for the description of
dispersion in a linear approximation.
The first is to use the reference momentum to calculate the quadrupole
focusing strength ($k_{0q} = \frac{gc}{p_0} \times 10^{-3}$) and to
include terms in the expressions for $x$, $x^\prime$, $y$, and
$y^\prime$ dependent on $\delta$.
The second is to use equation~\ref{Eq:Effectivekq} to calculate the
effective quadrupole focusing strength.
The second approach has been adopted here.

In the same notation, the transfer matrix for a quadrupole focusing in
the $yz$ plane (a ``defocusing quadrupole'') may be written: 
\begin{equation}
  \underline{\underline{T}}_{\rm~Dquad} =
    \begin{pmatrix}
          \cosh(\sqrt{k_q} l_q) & \frac{\sinh(\sqrt{k_q} l_q)}{\sqrt{k_q}} & 0 & 0             & 0 & 0 \\
 \sqrt{k_q} \sinh(\sqrt{k_q} l_q) &                  \cosh(\sqrt{k_q} l_q) & 0 & 0             & 0 & 0 \\
          0 & 0 &            \cos(\sqrt{k_q} l_q) &  \frac{\sin(\sqrt{k_q} l_q)}{\sqrt{k_q}} & 0 & 0 \\
          0 & 0 &  -\sqrt{k_q} \sin(\sqrt{k_q} l_q) &                   \cos(\sqrt{k_q} l_q) & 0 & 0 \\
          0 & 0 & 0 & 0 & 1 & \frac{l_q}{\beta_0^2 \gamma_0^2} \\
          0 & 0 & 0 & 0 & 0 &                             1
        \end{pmatrix} \, .
\end{equation}

\subsection{Solenoid}

The passage of a beam particle through a solenoid is determined by
the magnetic field strength, $\bm{B}_s$, within the solenoid and the
length of the solenoid, $l$, measured along its axis.
As the particle enters the solenoid, the fringe fields impart momentum
transverse to the axis of the magnet.
This results in the particle executing a helical trajectory, the axis
of the helix being parallel to the solenoid axis.
The sense of the rotation depends on the particle charge and the
polarity of the field.
The helical motion means that the evolution of the particle motion in
the $xz$ plane is coupled with the evolution of the particle motion in
the $yz$ plane.

In the ``hard-edge'' approximation, the magnetic field inside the
magnet is given by $\bm{B}_s = (0, 0, B_{s0})$, where the solenoid axis
lies along the $z_{\rm RPLC}$ axis.
The solenoid field strength parameter is then given by:
\begin{equation}
  k_s = \left[ \frac{B_{s0} c}{2p} \times 10^{-3} \right]^2\,; \label{Eq:Effectiveks}
\end{equation}
where $B_{s0}$ is measured in T, $p$ in MeV/c and $c$ in m/s.

The transfer matrix for passage of a positive particle through a
solenoid with field pointing in the positive $z_{\rm RPLC}$ direction
may may be written: 
\begin{equation}
  \underline{\underline{T}}_{\rm~Sol} =
    \begin{pmatrix}
                             \cos^2(\sqrt{k_s} l) &   \frac{1}{2\sqrt{k_s}} \sin(\sqrt{k_s} l) &         \frac{1}{2} \sin(2\sqrt{k_s} l) & \frac{1}{\sqrt{k_s}} \sin^2(\sqrt{k_s} l) & 0 & 0 \\
          -\frac{\sqrt{k_s}}{2} \sin(2\sqrt{k_s} l) &                     \cos^2(\sqrt{k_s} l) &           -\sqrt{k_s}\sin^2(\sqrt{k_s} l) &         \frac{1}{2} \sin(2\sqrt{k_s} l) & 0 & 0 \\
                 -\frac{1}{2} \sin(2\sqrt{k_s} l) & -\frac{1}{\sqrt{k_s}} \sin^2(\sqrt{k_s} l) &                    \cos^2(\sqrt{k_s} l) & \frac{1}{2\sqrt{k_s}} \sin(2\sqrt{k_s} l) & 0 & 0 \\
                    \sqrt{k_s} \sin^2(\sqrt{k_s} l) &         -\frac{1}{2} \sin(2\sqrt{k_s} l) & -\frac{\sqrt{k_s}}{2} \sin(2\sqrt{k_s} l) &                    \cos^2(\sqrt{k_s} l) & 0 & 0 \\
          0 & 0 & 0 & 0 & 1 & \frac{l}{\beta_0^2 \gamma_0^2} \\
          0 & 0 & 0 & 0 & 0 &                             1
        \end{pmatrix} \, .
\end{equation}
As in the case of the quadrupoles, dispersion is accounted for using
equation~\ref{Eq:Effectiveks}.

\subsection{Non-neutral (electron) plasma (Gabor) lens}

A dense gas of electrons confined in a Penning-Malmberg trap provides
an electric field that can be used to focus a positive ion beam.
The electron gas is confined axially in the lens by an electostatic
potential created using a central anode of length $l$.
The gas is confined radially using the uniform field of a solenoid.
Assuming a uniform electron density, $n_e$, the focusing parameter,
$k_G$, may be written:
\begin{equation}
  k_G = \frac{e}{2\epsilon_0} \frac{m_p \gamma}{p^2} n_e \,;
            \label{Eq:EffectivekG}
\end{equation}
where $e$ is the charge on the electron, $\epsilon_0$ is the
permittivity of free space, and $m_p$ is the proton mass.
As in the case of the quadrupoles and solenoid, dispersion is
accounted for using equation~\ref{Eq:EffectivekG}.
The force on a particle passing through the electron gas is towards
the axis of the lens and is proportional to the radia distance of the
particle from the axis.
Focusing is therefore cylindrically symmetric and does not couple
motion in the the $xz$ and $yz$ planes.

In the ``hard-edge'' approximation, the electric field inside the
lens falls to zero at the end of the electron gas and the contribution
of the magnetic field used to confine the electron gas in the
transverse direction has a negligible effect on particles passing
through hthe lens.
The transfer matrix for passage of a positive particle through the
lens may be written: 
\begin{equation}
  \underline{\underline{T}}_{G} =
    \begin{pmatrix}
                    \cos(\sqrt{k_G} l) & \frac{\sin(\sqrt{k_G} l)}{\sqrt{k_G}} &  0 & 0 & 0 & 0 \\
        -\sqrt{k_G} \sin(\sqrt{k_G} l) &                    \cos(\sqrt{k_G} l) &  0 & 0 & 0 & 0 \\
                                     0 &                                     0 &              \cos(\sqrt{k_G} l) & \frac{\sin(\sqrt{k_G} l)}{\sqrt{k_G}}  & 0 & 0 \\
                                     0 &                                     0 &  -\sqrt{k_G} \sin(\sqrt{k_G} l) &                     \cos(\sqrt{k_G} l) & 0 & 0 \\
          0 & 0 & 0 & 0 & 1 & \frac{l}{\beta_0^2 \gamma_0^2} \\
          0 & 0 & 0 & 0 & 0 &                             1
        \end{pmatrix} \, .
\end{equation}
