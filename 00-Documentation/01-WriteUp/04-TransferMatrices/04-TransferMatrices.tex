\graphicspath{ {04-TransferMatrices/Figures/} }

\section{Transfer matrices}

Description of beam transport often carried out by:
\begin{itemize}
  \item Breaking lattice down into a series of ``elements'',
    e.g. drift, quadrupole, dipole, solenoid, etc.;
  \item Transport of particle between the start and end of a
    particular element of the lattice linearised such that the trace
    space at the end of the element, $\bm{\phi}_f$, is written in
    terms of the trace space at the start of the element,
    $\bm{\phi}_i$:
    \begin{equation}
      \bm{\phi}_f = \underline{\underline{T}} ~ \bm{\phi}_i \,
      {\rm ; and}
    \end{equation}
  \item The distance along the reference particle trajectory is
    increased by $\delta s$, where:
    \begin{equation}
      s_{\rm end} = s_{\rm start} + \delta s \, ;
    \end{equation}
    and $s_{\rm start}$ and $s_{\rm end}$ are the length of the
    reference particle trajectory at the start and end of the 
    beam-line element respectively.
\end{itemize}
Many excellent descriptions of derivation of transfer matrices,
$\underline{\underline{T}}$, so only quote results here.

\subsection{Drift}

A ``drift'' space refers to a region in which the beam propagates in
the absence of any electromagnetic fields.
In a drift, particles propogate in straight lines, therefore:
\begin{equation}
  \underline{\underline{T}}_{\rm~drift} =
        \begin{pmatrix}
          1 & l & 0 & 0 & 0 &                             0 \\
          0 & 1 & 0 & 0 & 0 &                             0 \\
          0 & 0 & 1 & l & 0 &                             0 \\
          0 & 0 & 0 & 1 & 0 &                             0 \\
          0 & 0 & 0 & 0 & 1 & \frac{l}{\beta_0^2 \gamma_0^2} \\
          0 & 0 & 0 & 0 & 0 &                             1
        \end{pmatrix} \, ; 
\end{equation}
where $l$ is the length of the drift and
$\gamma_0=(1-\beta_0^2)^{-\frac{1}{2}}$.
The increment in the reference particle trajectory is:
\begin{equation}
  \delta s = l \, .
\end{equation}
