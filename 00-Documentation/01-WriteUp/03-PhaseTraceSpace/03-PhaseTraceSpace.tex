\graphicspath{ {03-PhaseTraceSpace/Figures/} }

\section{Phase space and trace space}

Para 1:
\begin{itemize}
  \item Quantum effects required to describe, e.g., development of
    polarisation in electron storage rings;
  \item Description of beam dynamics largely done using classical,
    Hamiltonian, mechanics;
  \item In classical mechanics, the equations of motion are solved to
    give the evolution of the position, momentum, and energy of a
    particle as a function of time;
  \item Check Goldstein, time as parameter.
\end{itemize}

Para 2:
\begin{itemize}
  \item Relativistic mechanics exploits four-vector position
    ($\underline{R} = (\bm{r}, ct)$) and four-vector momentum
    ($\underline{P} = (\bm{p}, E)$); 
  \item Hamiltonian mechanics uses time as a paramter, i.e.
    $\underline{R} = \underline{R}(t)$ and
    $\underline{P} = \underline{P}(t)$;
  \item Time, or a quantity directly related to time such as the path
    length, $s$, is therefore a paramter.
    In addition, $E$ and $\bm{p}$ are related by the invariant mass of
    particle.
\end{itemize}

Para 3:
\begin{itemize}
  \item Phase space is the position of the particle in coordinate and
    and momentun space;
  \item Six phase-space coordinates are required and are usually to
    taken to be $\bm{r}$ and $\bm{p}$;
  \item This section defines the 6-dimensional spaces used to describe
    particle phase space in the linear optics code.
\end{itemize}

\subsection{Phase space}

The 6D phase-space vector is defined in terms of the three-vector
position and three vector momentum as:
\begin{equation}
  \begin{bmatrix} \bm{r} \\ \bm{p} \end{bmatrix} = 
  \begin{bmatrix} \begin{pmatrix} x   \\   y \\ z   \end{pmatrix} \\
                  \begin{pmatrix} p_x \\ p_y \\ p_z \end{pmatrix} \end{bmatrix}
\end{equation}
The trajectory of the particle may be evaluated as a function of time
or $s$.

\subsection{Trace space}

Trace space is defined to simplify the calculation of the trajectory
of particles through the accelerator lattice and is derived from the
phase space expressed in the RPLC frame.
Consider a particle with position
$(x_{\rm\,RPLC}, y_{\rm\,RPLC}, z_{\rm\,RPLC})$ and momentum 
$p_{\rm\,RPLC}$ with components
$(p_{\rm x\,RPLC}, p_{\rm y\,RPLC}, p_{\rm z\,RPLC})$.
Taking the momentum of the reference particle in the laboratory frame
to be $p_0$, the trace-space coordinates are given by:
\begin{equation}
    \bm{\phi} = \begin{pmatrix}
                  x_{\rm\,RPLC}       \\
                  x\prime_{\rm\,RPLC} \\
                  y_{\rm\,RPLC}       \\
                  y\prime_{\rm\,RPLC} \\
                  z_{\rm\,RPLC}       \\
                  \delta_{\rm\,RPLC}
                \end{pmatrix} \, ;
\end{equation}
where:
\begin{eqnarray}
  x\prime_{\rm\,RPLC}  & = & \frac{\partial x}{\partial s} = \frac{cp_{x\,{\rm\,RPLC}}}{cp_0} \, ; \\
  y\prime_{\rm\,RPLC}  & = & \frac{\partial y}{\partial s} = \frac{cp_{y\,{\rm\,RPLC}}}{cp_0} \, ; \\
  z_{\rm\,RPLC}        & = & \frac{s}{\beta_0} - ct = \frac{\Delta s}{\beta_0}                    \, ; \\
  \delta_{\rm\,RPLC}   & = & \frac{E}{c p_0} - \frac{1}{\beta_0} = \frac{\Delta E}{cp_0}          \, {\rm ; and}
\end{eqnarray}
$\Delta s$ and $\Delta E$ are the differences between the reference
particle position, $s_0$, and its energy, $E_0$, and the position and
energy of the test particle, $s$ and $E$ respectively.  
$\Delta s$ and $\Delta E$ are given by $\Delta s = s - s_0$ and
$\Delta E = E - E_0$.


