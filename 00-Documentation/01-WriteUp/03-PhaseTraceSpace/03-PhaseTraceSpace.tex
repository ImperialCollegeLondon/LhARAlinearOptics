\graphicspath{ {03-PhaseTraceSpace/Figures/} }

\section{Phase space, trace space, beam parameters}

The motion of particles passing through an accelerator is most often
described using classical Hamiltonian mechanics; quantum mechanics
being required only in particular cases such as the description of
spin polarisation in a storage ring.
In classical Hamiltonian mechanics the equations of motion are solved
to give the evolution of the position, momentum, and energy as
functions of a single independent parameter.
The independent parameter is often taken to be time.

Relativistic mechanics exploits four-vector position,
$\underline{\cal{R}} = (\bm{r}, ct)$, and four-vector momentum,
$\underline{\cal{P}} = (c\bm{p}, E)$.
In the Hamiltonian description of particle dynamics, these four
vectors become functions of the independent variable, i.e.
$\underline{\cal{R}} = \underline{\cal{R}}(t)$ and
$\underline{\cal{P}} = \underline{\cal{P}}(t)$.
In the laboratory system, the position of the reference particle along
its trajectory is directly related to the time coordinate by
$t = c \beta_0 s$.
This allows $s$ to be taken as the independent variable and for the
motion of particles in the beam to be derived as functions of $s$.

The 6D phase-space coordinates of a particle as a function of $s$ are
given by the position and momentum three vectors.
The particle energy may be determined from the invariant mass and the
time coordinate from the invariant interval between the origin and the
position represented by $s$.

The ``trace-space'' coordinates of a particle are defined relative to
the reference particle.
Usually, a beam is understood to contain particles which follow
trajectories that differ rather little from that of the reference
particle.
Trace space is defined such that the position, ``momentum'', and
``energy'' coordinates are small for particles which follow
trajectories close to that of the reference particle.
The utility of this approach is that trace-space coordinates may be
used to perform Taylor expansions of the Hamiltonian which may readily
be solved to yield a description of particle transport using functions
that are linear in the trace-space coordinates.

The notation used for the 6D phase and trace spaces are defined in
this section.

\subsection{Phase space}

The 6D phase-space vector is defined in terms of the three-vector
position and three vector momentum as:
\begin{equation}
  \begin{bmatrix} \bm{r} \\ \bm{p} \end{bmatrix} = 
  \begin{bmatrix} \begin{pmatrix} x   \\   y \\ z   \end{pmatrix} \\
                  \begin{pmatrix} p_x \\ p_y \\ p_z \end{pmatrix} \end{bmatrix}
\end{equation}
The trajectory of the particle may be evaluated as a function of time
or $s$.

\subsection{Trace space}

Trace space is defined to simplify the calculation of the trajectory
of particles through the accelerator lattice and is derived from the
phase space expressed in the RPLC frame.
Consider a particle with position
$\bm{r}_{\rm\,RPLC}=(x_{\rm\,RPLC}, y_{\rm\,RPLC}, z_{\rm\,RPLC})$ and
momentum
$\bm{p}_{\rm\,RPLC} = (p_{\rm x\,RPLC}, p_{\rm y\,RPLC}, p_{\rm z\,RPLC})$.
Taking the magnitude of the momentum of the reference particle in the
laboratory frame to be $p_0$, the trace-space coordinates are given
by:
\begin{equation}
    \underline{\phi} = \begin{pmatrix}
                        x_{\rm\,RPLC}       \\
                        x^\prime_{\rm\,RPLC} \\
                        y_{\rm\,RPLC}       \\
                        y^\prime_{\rm\,RPLC} \\
                        z_{\rm\,RPLC}       \\
                        \delta_{\rm\,RPLC}
                      \end{pmatrix} \, ;
\end{equation}
where:
\begin{eqnarray}
  x^\prime_{\rm\,RPLC}  & = & \frac{\partial x}{\partial s}
                        = \frac{cp_{x\,{\rm\,RPLC}}}{cp_0} \, ; \\
  y^\prime_{\rm\,RPLC}  & = & \frac{\partial y}{\partial s}
                        = \frac{cp_{y\,{\rm\,RPLC}}}{cp_0} \, ; \\
  z_{\rm\,RPLC}        & = & \frac{s}{\beta_0} - ct
                       = \frac{\Delta s}{\beta_0}             \, {\rm ; and} \\
  \delta_{\rm\,RPLC}   & = & \frac{E}{c p_0} - \frac{1}{\beta_0}
                        = \frac{\Delta E}{cp_0}                   \, .    
\end{eqnarray}
Here $\Delta s = s - s_0$ and $\Delta E = E - E_0$, where $s_0$ and
$E_0$ are the reference particle position and energy respectively; $E$
and $s$ are the energy and position of a particular particle in the
beam.

\subsection{Beam parameters}

The trace space vector, $\underline{\phi}_i$, of the $i^{\rm th}$
particle at a position along the beam line contains the variance of
the particle coordinates with respect to those of the reference
particle.
For a sample containing $N$ particles, the covariance matrix,
$\underline{\underline{C}}_6$, may therefore be obtained by evaluating:
\begin{equation}
  \underline{\underline{C}}_6 = \frac{1}{N} \sum_i^N
                                    \underline{\phi}_i \; \underline{\phi}_i^T
                              = \left<
                                    \underline{\phi} \; \underline{\phi}^T
                                \right> \, ;
  \label{Eq:Cov6}
\end{equation}
where the notation $\left< \right>$ is used to denote evaluating the
expectation value.
The RMS emittance of the beam in all 6 trace-space dimensions is then
given by:
\begin{equation}
  \varepsilon_6 = \sqrt[6]{\left| \underline{\underline{C}}_6 \right| }\;.
\end{equation}
In an analagous notation, the four-dimensional transverse trace space
$(x, x^\prime, y, y^\prime)$ may be used to define the
four-dimensional covariance matrix $\underline{\underline{C}}_4$,
which, in turn, can be used to evaluate the four-dimensional
transverse emittance:
\begin{equation}
  \varepsilon_4 = \sqrt[4]{\left| \underline{\underline{C}}_4 \right| }\;.
\end{equation}
The size of the beam in the two ``transverse planes'' $(x, x^\prime)$
and $(y, y^\prime)$ is contained in two, $2 \times 2$, submatrices of
$\underline{\underline{C}}_6$, $\underline{\underline{C}}_x$ and
$\underline{\underline{C}}_y$.
If the $x$ and $y$ coordinates are to be taken as uncoupled, then the
emittance of the $(x, x^\prime)$ and $(y, y^\prime)$ trace space my
be obtained using:
\begin{eqnarray}
  \varepsilon_x & = & \sqrt     {\left| \underline{\underline{C}}_x \right| }
    {\rm~and} \\
  \varepsilon_y & = & \sqrt{\left| \underline{\underline{C}}_y \right| }\;.
\end{eqnarray}
With $u=x~{\rm or}~y$, the area of the trace space ellipse is given
by $\pi \varepsilon_u$ and the Twiss parameters, $\alpha_u$,
$\beta_u$, and $\gamma_u$ are given by:
\begin{eqnarray}
  \sigma_u^2 = \left< u^2 \right> &         = & \beta_u, \varepsilon_u \, ;            \\
  \left< u^{\prime 2} \right> &              = & \gamma_u, \varepsilon_u \, {\rm : and} \\
  \left< u u^\prime \right>   &              = & -\alpha_u \varepsilon_u \, ;
\end{eqnarray}
where:
\begin{equation}
  \beta_u \gamma_u - \alpha_u^2 = 1 \, .
\end{equation}

