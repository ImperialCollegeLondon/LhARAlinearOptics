\graphicspath{ {06-MCD/Figures/} }

\section{Module, class and data structures}

The linear optics package has been written in object-oriented Python
and is broken down in four principal modules:
\begin{itemize}
  \item \underline{\texttt{BeamLineElement}:} provides descriptions
    of the various beam-line elements required to build a description
    of the beam line.
    Each individual element, such as a drift, quadrupole, etc., is
    described in a class derived from the \texttt{BeamLineElement}
    parent class.
  \item \underline{\texttt{BeamLine}:} provides code to assemble the
    elements into a coherent beam line.
    \texttt{BeamLine} is a singleton class to ensure that two beam
    lines can not be simulated in a single run of the package.
  \item \underline{\texttt{Beam}:} provides code to calculate
    ensemble properties of the beam such as emittance, etc.
    The ensemble properties are stored as instance attributes of
    the \texttt{Beam} class.
  \item \underline{\texttt{Particle}:} provides code to record beam
    particles at positions along the beam line.
    The module provides the singleton \texttt{ReferenceParticle}
    class derived from the \texttt{Particle} class.
\end{itemize}
Other modules: \texttt{BeamIO}, \texttt{LaTeX},
\texttt{PhysicalConstants}, \texttt{Report}, \texttt{Simulation}
and \texttt{Utilities} support the principal modules or provide
services.
The data structure is implemented as attributes of the instances of the 
various classes.
This section describes the implementation of the various modules, the
classes of which they are composed, and how access to the data is
provided.

Each class has methods by which to access a list of the class
instances and a Boolean flag by which to generate debug print out (see
table~\ref{Tab:ClassAttributeAccess}).
\begin{table}[h]
  \caption{Methods by which to set and access class attributes.}
  \label{Tab:ClassAttributeAccess}
  \begin{center}
    \begin{tabular}{|l|c|l|p{7cm}|}
      \hline
      \textbf{Method}          & \textbf{Argument} & \textbf{Return}            & \textbf{Comment}                      \\
      \hline
      \texttt{getinstances()}  &                   & List of instances of class &                                       \\
      \texttt{setDebug(Debug)} & Boolean           &                            & Sets flag to generate debug print-out \\
      \texttt{getDebug()}      &                   & Boolean debug flag         & If true, generate debug print-out     \\
      \texttt{setAll2None()}   &                   &                            & Set all instance attributes to \texttt{None} at start of instantiation. \\
      \hline
    \end{tabular}
  \end{center}
\end{table}

\subsection{\texttt{BeamLineElement}}

\subsubsection{Parent class}

\paragraph{Instantiation} ~\newline
\noindent
The call to instantiate the \texttt{BeamLineElenent} class is:
\begin{center}
  \texttt{BeamLineElement(Name, rStrt, vStrt, drStrt)}
\end{center}
The arguments are translated into instance attributes as described in
section~\ref{SubSubSect:BLE:InstAttr} and defined in
table~\ref{Tab:BLE:Attributes}.

\paragraph{Instance attributes and access methods} ~\newline
\label{SubSubSect:BLE:InstAttr}
\noindent
Properties common to all beam-line elements are stored as instance
attributes of the parent \texttt{BeamLineElement} class.
The instance attributes are defined in table~\ref{Tab:BLE:Attributes}.
The attributes are accessed and set using the methods defined in
table~\ref{Tab:BLE:Methods}.
\begin{table}[h]
  \caption{
    Definition of attributes of instances of
    the \texttt{BeamLineElement} class.
    The attributes marked $^*$ above the dividing line are required in
    the call to instantiate the element.
    The attributes marked $^\dagger$ below the dividing line are
    calculated.
  }
  \label{Tab:BLE:Attributes}
  \begin{center}
    \begin{tabular}{|l|c|c|p{10cm}|}
      \hline
      \textbf{Attribute} & \textbf{Type} & \textbf{Unit} & \textbf{Comment}                                                                   \\
      \hline
      \texttt{Name}$^*$     & String        &     & Name of beam-line element.                                                                \\
      \texttt{rStrt}$^*$    & numpy.ndarray & m   & $[x, y, z]$ position of entrance to element in laboratory coordinate system.              \\
      \texttt{vStrt}$^*$    & numpy.ndarray & rad & $[[i],[\theta, \phi]]$ (polar and azimuthal angles) of RPLC $y$ and $z$ axes ($i=0,1$ respectively) at start. \\
      \texttt{drStrt}$^*$   & numpy.ndarray & m   & ``Error'', $[x, y, z]$, displacement of start from nominal position (not yet implemented).\\
      \texttt{dvStrt}$^*$   & numpy.ndarray & rad & ``Error'', $[[i],[\theta, \phi]]$, deviation in $\theta$ and $\phi$ from nominal axis (not yet implemented). \\
      \hline
      \texttt{Strt2End}$^\dagger$     & numpy.ndarray &     & $1\times3$ translation from start of element to end; in laboratory coordinates.  Set in derived class. \\
      \texttt{Rot2LbStrt}$^\dagger$   & numpy.ndarray &     & $3\times3$ rotation matrix that takes RPLC axes to laboratory axes at start.     \\
      \texttt{Rot2LbEnd}$^\dagger$    & numpy.ndarray &     & $3\times3$ rotation matrix that takes RPLC axes to laboratory axes at end.  Set in derived class.      \\
      \texttt{TnrsMtrx}$^\dagger$     & numpy.ndarray &     & $3\times3$ transfer matrix.  Set in derived class.                               \\
      \hline
    \end{tabular}
  \end{center}
\end{table}
\begin{table}[h]
  \caption{
    Definition of access methods for the \texttt{BeamLineElement}
    class. 
  }
  \label{Tab:BLE:Methods}
  \begin{center}
    \begin{tabular}{|c|c|p{7cm}|}
      \hline
      \textbf{Set method} & \textbf{Get method}  & \textbf{Comment}                                                         \\
      \hline
      \texttt{setName(Name)}     & \texttt{getName()}        & Set/get name of beam-line element.                                \\
      \texttt{setrStrt(rStrt)}   & \texttt{getrStrt()}       & Set/get laboratory $[x, y, z]$ position of entrance.              \\
                                         \texttt{setvStrt(vStrt)}   & \texttt{getvStrt()}      & Set/get RPLC $[\theta, \phi]$ of principal axis.                  \\
      \texttt{setdrStrt(drStrt)} & \texttt{getdrStrt()}      & Set/get ``error'' displacement.                                   \\
      \texttt{setdvStrt(dvStrt)} & \texttt{getdvStrt()}      & Set/get ``error'' deviation in $[\theta, \phi]$.                  \\
      \texttt{setRot2LbStrt()}   & \texttt{getRot2LbStrt()}  & Set/get rotation matrix from RPLC axes to laboratory.             \\
      \texttt{setRot2LabStrt()}  & \texttt{getRot2LbStrt()}  & Setget rotation matrix from RPLC to laboratory at start.             \\
      \texttt{setStrt2End(t)}    & \texttt{getStrt2End()}    & Set/get displacement vector start to end in laboratory coordinates.
                                                               setStrt2End takes 1 argument, \texttt{t}, a 1D np.ndarray containing
                                                               the translation from the start to the end of the element in RPLC. \\
      \texttt{setRot2LbEnd(R)}  & \texttt{getRot2LbEnd()}    & Set/get rotation matrix from RPLC to laboratory at end.
                                                               setRot2LbEnd takes 1 argument, \texttt{R}, a 2D np.array containing
                                                               the rotation matrix to be set.                                    \\
                                & \texttt{getTransferMatrix()}  & Get transfer matrix set in derived class.                         \\
      \hline
    \end{tabular}
  \end{center}
\end{table}

\paragraph{Processing methods} ~\newline
\noindent
Table~\ref{Tab:BLE:ProcMethods} presents the processing methods provided
in the \texttt{BeamLineElement} class.
\begin{table}[h]
  \caption{
    Processing methods provided by the \texttt{BeamLineElement}
    class. 
  }
  \label{Tab:BLE:ProcMethods}
  \begin{center}
    \begin{tabular}{|l|c|c|p{7cm}|}
      \hline
      \textbf{Method} & \textbf{Argument(s)} & \textbf{Return} & \textbf{Comment}                                            \\
      \hline
      \texttt{OutsideBeamPipe(R)} & Float                 & Boolean               & Returns False if particle is inside beam pipe.
                                                                                    If \texttt{R}, radial distance from $z$ axis in RPLC,
                                                                                    falls outside beam pipe, returns True. \\
      \texttt{Transport(V)}       & $6\times1$ np.ndarray & $6\times1$ np.ndarray & Transport 6D trace-space vector, \texttt{V}, across
                                                                                    element.
                                                                                    Final trace-space vector returned.       \\
      \texttt{Shit2Local(V)}      & $6\times1$ np.ndarray & $6\times1$ np.ndarray & Transform 6D trace-space vector, \texttt{V}, from RPLC
                                                                                    to laboratory coordinates.
                                                                                    Phase-space vector in laboratory frame returned. \\
      \texttt{Shit2Laboratory(U)} & $6\times1$ np.ndarray & $6\times1$ np.ndarray & Transform 6D phase-space vector, \texttt{U}, from 
                                                                                    laboratory coordinates to trace-space coordinates in
                                                                                    the RPLC frame.
                                                                                    Trace-space vector in RLPC frame returned. \\
      \hline
    \end{tabular}
  \end{center}
\end{table}

\paragraph{I/o methods} ~\newline
\noindent
Methods to read and write instance attributes to the files defined
using the \texttt{BeamIO} package (see section \ref{Sect:BeamIO} are
provided.
The calls are:
\begin{equation}
  \texttt{readElement(dataFILE)} \quad \quad {\rm and }
      \quad \quad \texttt{writeElement(dataFILE)} \,; \nonumber
\end{equation}
where \texttt{dataFILE} is the a file instance managed by \texttt{BeamIO}.

\paragraph{Utilities} ~\newline
\noindent
Table~\ref{Tab:BLE:Utilities} presents the utilities provided in the
\texttt{BeamLineElement} class.
\begin{table}[h]
  \caption{
    Utilities provided by the \texttt{BeamLineElement}
    class. 
  }
  \label{Tab:BLE:Utilities}
  \begin{center}
    \begin{tabular}{|l|c|c|p{7cm}|}
      \hline
      \textbf{Method} & \textbf{Argument(s)} & \textbf{Return} & \textbf{Comment}                                            \\
      \hline
      \texttt{cleaninstances()}     &                 &  & Delete (using ``del'') all instances of the \texttt{BeamLineElement} class.
                                                           Reset \texttt{instances} list. \\
      \texttt{removeInstance(inst)} & Instance of BLE &  & Remove instance \texttt{inst} and remove from list of instances of
                                                           \texttt{BeamLineElement}. \\
      \hline
    \end{tabular}
  \end{center}
\end{table}

\FloatBarrier

\subsubsection{Derived class: \texttt{Facility(BeamLineElement)}}

\paragraph{Instantiation} ~\newline
\noindent
The call to instantiate the \texttt{Facility} derived class is:
\begin{center}
  \texttt{FacilityName, rStrt, vStrt, drStrt, dvStrt, p0, VCMV)}
\end{center}
Parent class arguments \texttt{Name}, \texttt{rStrt}, \texttt{vStrt},
\texttt{drStrt}, and \texttt{dvStrt} are described in
section~\ref{SubSubSect:BLE:InstAttr}.
These arguments are passed directly to \texttt{BeamLineElement}.
The \texttt{Facility} arguments are translated into instance
attributes as described in section~\ref{SubSubSect:Fclty:InstAttr} and
defined in table~\ref{Tab:Fclty:Attributes}.

\paragraph{Instance attributes and access methods} ~\newline
\label{SubSubSect:Fclty:InstAttr}
\noindent
The instance attributes are defined in
table~\ref{Tab:Fclty:Attributes}. 
The attributes are accessed and set using the methods defined in
table~\ref{Tab:Fclty:Methods}.
\begin{table}[h]
  \caption{
    Definition of attributes of instances of
    the \texttt{Facility(BeamLineElement)} derived class.
    All attributes are required in the call to instantiate the
    element.
  }
  \label{Tab:Fclty:Attributes}
  \begin{center}
    \begin{tabular}{|l|c|c|p{10cm}|}
      \hline
      \textbf{Attribute} & \textbf{Type} & \textbf{Unit} & \textbf{Comment}                                                                   \\
      \hline
      \texttt{p0}   & float & MeV & Kinetic energy of reference particle. \\
      \texttt{VCMV} & float & m   & Radius of vacuum-chamber mother
                                    volume.
                                    The radius defines edge of the volume
                                    at which a particle trajectory is
                                    terminated.
                                    It may be necessary to introduce a beam
                                    pipe later.                            \\
      \hline
    \end{tabular}
  \end{center}
\end{table}
\begin{table}[h]
  \caption{
    Definition of access methods for the \texttt{Facility} derived
    class. 
  }
  \label{Tab:Fclty:Methods}
  \begin{center}
    \begin{tabular}{|c|c|p{7cm}|}
      \hline
      \textbf{Set method} & \textbf{Get method}  & \textbf{Comment}                                   \\
      \hline
      \texttt{setp0(Name)}   & \texttt{getp0()}    & Set/get momentum of reference particle (in MeV). \\
      \texttt{setVCMV(VCMV)} & \texttt{getrVCMV()} & Set/get radius of vacuum chamber mother volume.  \\
      \hline
    \end{tabular}
  \end{center}
\end{table}

\paragraph{Processing methods} ~\newline
\noindent
The \texttt{Facility} derived class has no processing methods.

\paragraph{I/o methods} ~\newline
\noindent
Methods to read and write instance attributes to the files defined
using the \texttt{BeamIO} package (see section \ref{Sect:BeamIO} are
provided.
The calls are:
\begin{equation}
  \texttt{readElement(dataFILE)} \quad \quad {\rm and }
      \quad \quad \texttt{writeElement(dataFILE)} \,; \nonumber
\end{equation}
where \texttt{dataFILE} is the a file instance managed by \texttt{BeamIO}.


\paragraph{Utilities} ~\newline
\noindent
The \texttt{Facility} derived class has no utilities.

\FloatBarrier

\subsubsection{Derived class: \texttt{Drift(BeamLineElement)}}

\paragraph{Instantiation} ~\newline
\noindent
The call to instantiate the \texttt{Drift} derived class is:
\begin{center}
  \texttt{Drift(Name, rStrt, vStrt, drStrt, dvStrt, Length)}
\end{center}
Parent class arguments \texttt{Name}, \texttt{rStrt}, \texttt{vStrt},
\texttt{drStrt}, and \texttt{dvStrt} are described in
section~\ref{SubSubSect:BLE:InstAttr}.
These arguments are passed directly to \texttt{BeamLineElement}.
The \texttt{Drift} arguments are translated into instance
attributes as described in section~\ref{SubSubSect:Drft:InstAttr} and
defined in table~\ref{Tab:Drft:Attributes}.

\paragraph{Instance attributes and access methods} ~\newline
\label{SubSubSect:Drft:InstAttr}
\noindent
The instance attributes are defined in
table~\ref{Tab:Drft:Attributes}. 
The attributes are accessed and set using the methods defined in
table~\ref{Tab:Drft:Methods}.
\begin{table}[h]
  \caption{
    Definition of attributes of instances of
    the \texttt{Drift(BeamLineElement)} derived class.
    All attributes are required in the call to instantiate the
    element.
  }
  \label{Tab:Drft:Attributes}
  \begin{center}
    \begin{tabular}{|l|c|c|p{10cm}|}
      \hline
      \textbf{Attribute} & \textbf{Type} & \textbf{Unit} & \textbf{Comment}                                                                   \\
      \hline
      \texttt{Length} & float & m & Length of drift. \\
      \hline
    \end{tabular}
  \end{center}
\end{table}
\begin{table}[h]
  \caption{
    Definition of access methods for the \texttt{Facility} derived
    class. 
  }
  \label{Tab:Drft:Methods}
  \begin{center}
    \begin{tabular}{|c|c|p{7cm}|}
      \hline
      \textbf{Set method} & \textbf{Get method}  & \textbf{Comment}                       \\
      \hline
      \texttt{setLength(Length)}   & \texttt{getLength()} & Set/get length of drift (in m). \\
      \texttt{setTransferMatrix()} &                      & Set transfer matrix.            \\
      \hline
    \end{tabular}
  \end{center}
\end{table}

\paragraph{Processing methods} ~\newline
\noindent
The \texttt{Facility} derived class has no processing methods.

\paragraph{I/o methods} ~\newline
\noindent
Methods to read and write instance attributes to the files defined
using the \texttt{BeamIO} package (see section \ref{Sect:BeamIO} are
provided.
The calls are:
\begin{equation}
  \texttt{readElement(dataFILE)} \quad \quad {\rm and }
      \quad \quad \texttt{writeElement(dataFILE)} \,; \nonumber
\end{equation}
where \texttt{dataFILE} is the a file instance managed by \texttt{BeamIO}.

\paragraph{I/o methods} ~\newline
\noindent
The \texttt{Facility} derived class has no utilities.

\FloatBarrier

\subsubsection{Derived class: \texttt{Aperture(BeamLineElement)}}

\paragraph{Instantiation} ~\newline
\noindent
The call to instantiate the \texttt{Aperture} derived class is:
\begin{center}
  \texttt{Aperture(Name, rStrt, vStrt, drStrt, dvStrt, ParamList)}
\end{center}
Parent class arguments \texttt{Name}, \texttt{rStrt}, \texttt{vStrt},
\texttt{drStrt}, and \texttt{dvStrt} are described in
section~\ref{SubSubSect:BLE:InstAttr}.
These arguments are passed directly to \texttt{BeamLineElement}.
The \texttt{Aperture} arguments are translated into instance
attributes as described in section~\ref{SubSubSect:Aprtr:InstAttr} and
defined in table~\ref{Tab:Aprtr:Attributes}.

\paragraph{Instance attributes and access methods} ~\newline
\label{SubSubSect:Aprtr:InstAttr}
\noindent
The instance attributes are defined in
table~\ref{Tab:Aprtr:Attributes}. 
The attributes are accessed and set using the methods defined in
table~\ref{Tab:Aprtr:Methods}.
\begin{table}[h]
  \caption{
    Definition of attributes of instances of
    the \texttt{Aperture(BeamLineElement)} derived class.
    All attributes are required in the call to instantiate the
    element.
  }
  \label{Tab:Aprtr:Attributes}
  \begin{center}
    \begin{tabular}{|l|c|c|p{10cm}|}
      \hline
      \textbf{Attribute}   & \textbf{Type} & \textbf{Unit} & \textbf{Comment}                                                                   \\
      \hline
      \texttt{ParamList}   & [] &   & List containing aperture parameters.
                                    The first parameter is an \texttt{int}
                                    and defines the aperture ``\texttt{Type}''.
                                    The remaining elements in the parameter list
                                    are \texttt{float}s with meanings that depend
                                    on \texttt{Type}.                             \\
      \hline
      \texttt{ParamList[0]} & int   &   & \texttt{Type}$=0$; circular                 \\
      \texttt{ParamList[1]} & float & m & Radius of circular aperture                 \\
      \hline
      \texttt{ParamList[0]} & int   &   & \texttt{Type}$=1$; Elliptical                           \\
      \texttt{ParamList[1]} & float & m & Radius of elliptical aperture along $x_{\rm RPLC}$ axis \\
      \texttt{ParamList[2]} & float & m & Radius of elliptical aperture along $y_{\rm RPLC}$ axis \\
      \hline
    \end{tabular}
  \end{center}
\end{table}
\begin{table}[h]
  \caption{
    Definition of access methods for the \texttt{Facility} derived
    class. 
  }
  \label{Tab:Aprtr:Methods}
  \begin{center}
    \begin{tabular}{|c|c|p{7cm}|}
      \hline
      \textbf{Set method} & \textbf{Get method}  & \textbf{Comment}                                  \\
      \hline
      \texttt{setApertureParameters(ParamList)} &                      & Set aperture paramters.
                                                                         Sets \texttt{Type} and parameters
                                                                         depending on \texttt{Type}. \\
                                                & \texttt{getType()} & Get \texttt{Type} of aperture.\\
                                                & \texttt{getParams()} & Get aperture parameters.    \\
      \texttt{setTransferMatrix()}              &                      & Set transfer matrix.        \\
      \hline
    \end{tabular}
  \end{center}
\end{table}

\paragraph{Processing methods} ~\newline
\noindent
The \texttt{Aperture} processing method is defined in
table~\ref{Tab:Aprtr:Methods}.
\begin{table}[h]
  \caption{
    Utilities provided by the \texttt{Aperture} derived
    class. 
  }
  \label{Tab:Aprtr:Methods}
  \begin{center}
    \begin{tabular}{|l|c|c|p{7cm}|}
      \hline
      \textbf{Method} & \textbf{Argument(s)} & \textbf{Return} & \textbf{Comment}                                            \\
      \hline
      \texttt{Transport(V)} & \texttt{np.ndarray} & \texttt{np.ndarray} or \texttt{None} &
                        Transport trace-space vector \texttt{V}.  If \texttt{V} falls outside of the aperture, return \texttt{None}. \\
      \hline
    \end{tabular}
  \end{center}
\end{table}

\paragraph{I/o methods} ~\newline
\noindent
Methods to read and write instance attributes to the files defined
using the \texttt{BeamIO} package (see section \ref{Sect:BeamIO} are
provided.
The calls are:
\begin{equation}
  \texttt{readElement(dataFILE)} \quad \quad {\rm and }
      \quad \quad \texttt{writeElement(dataFILE)} \,; \nonumber
\end{equation}
where \texttt{dataFILE} is the a file instance managed by \texttt{BeamIO}.

\paragraph{Utilities} ~\newline
\noindent
The \texttt{Aperture} derived class has no utilities.

\FloatBarrier

\subsubsection{Derived class: \texttt{FocusQuadrupole(BeamLineElement)}}

\paragraph{Instantiation} ~\newline
\noindent
The call to instantiate the \texttt{FocusQuadrupole} derived class is:
\begin{center}
  \texttt{FocusQuadrupole(Name, rStrt, vStrt, drStrt, dvStrt, Length,
          Strength, kFQ)} 
\end{center}
Parent class arguments \texttt{Name}, \texttt{rStrt}, \texttt{vStrt},
\texttt{drStrt}, and \texttt{dvStrt} are described in
section~\ref{SubSubSect:BLE:InstAttr}.
These arguments are passed directly to \texttt{BeamLineElement}.
The \texttt{FocusQuadrupole} arguments are translated into instance
attributes as described in section~\ref{SubSubSect:FQuad:InstAttr} and
defined in table~\ref{Tab:FQuad:Attributes}.
The quadrupole \texttt{Length} is required together with
either the field gradient, \texttt{Strength}
(equation~\ref{Eq:Trnsf:gxy}), or the quadrupole $k$
parameter, \texttt{kFQ} (equation~\ref{Eq:Effectivekq}).

\paragraph{Instance attributes and access methods} ~\newline
\label{SubSubSect:FQuad:InstAttr}
\noindent
The instance attributes are defined in
table~\ref{Tab:FQuad:Attributes}. 
The attributes are accessed and set using the methods defined in
table~\ref{Tab:FQuad:Methods}.
\begin{table}[h]
  \caption{
    Definition of attributes of instances of
    the \texttt{FocusQuadrupole(BeamLineElement)} derived class.
    All attributes are required in the call to instantiate the
    element.
  }
  \label{Tab:FQuad:Attributes}
  \begin{center}
    \begin{tabular}{|l|c|c|p{10cm}|}
      \hline
      \textbf{Attribute}   & \textbf{Type} & \textbf{Unit} & \textbf{Comment}                    \\
      \hline
      \texttt{FQmode}   & int   &       & If 0, use particle momentum in calculation of transfer
                                          matrix; if 1, use reference particle momentum.         \\
      \texttt{Length}   & float & m     & Effective length of quadrupole.                        \\
      \texttt{Strength} & float & T/m   & Magnetic field gradient; required if kFQ is not given. \\
      \texttt{kFQ}      & float & m$^2$ & Quadrupole $k$ paramter.                               \\
      \hline
    \end{tabular}
  \end{center}
\end{table}
\begin{table}[h]
  \caption{
    Definition of access methods for the \texttt{FocusQuadrupole} derived
    class. 
  }
  \label{Tab:FQuad:Methods}
  \begin{center}
    \begin{tabular}{|c|c|p{7cm}|}
      \hline
      \textbf{Set method} & \textbf{Get method}  & \textbf{Comment}                                    \\
      \hline
      \texttt{setFQmode(FQmode)}   & \texttt{getFQmode()}   & Set/get FQmode.                        \\
      \texttt{setLength(Length)}   & \texttt{getLength()}   & Set/get length.                        \\
      \texttt{setStrength(Length)} & \texttt{getStrength()} & Set/get strength (field gradient).     \\
      \texttt{setKFQ(Length)}      & \texttt{getKFQ()}      & Set/get kFQ, quadrupole $k$ parameter. \\
      \texttt{setTransferMatrix()} &                        & Set transfer matrix.                   \\
      \hline
    \end{tabular}
  \end{center}
\end{table}

\paragraph{Processing methods} ~\newline
\noindent
The \texttt{FocusQuadrupole} processing method is defined in
table~\ref{Tab:FQuad:Methods}.
\begin{table}[h]
  \caption{
    Utilities provided by the \texttt{FocusQuadrupole} derived
    class. 
  }
  \label{Tab:FQuad:Methods}
  \begin{center}
    \begin{tabular}{|l|c|c|p{7cm}|}
      \hline
      \textbf{Method} & \textbf{Argument(s)} & \textbf{Return} & \textbf{Comment}                     \\
      \hline
      \texttt{calckFQ()} &  & float & Calculates \texttt{kFQ} if strength is specified.               \\
      \texttt{calcStrength()} &  & float & Calculates \texttt{Strength} if \texttt{kFQ} is specified. \\
      \hline
    \end{tabular}
  \end{center}
\end{table}

\paragraph{I/o methods} ~\newline
\noindent
Methods to read and write instance attributes to the files defined
using the \texttt{BeamIO} package (see section \ref{Sect:BeamIO} are
provided.
The calls are:
\begin{equation}
  \texttt{readElement(dataFILE)} \quad \quad {\rm and }
      \quad \quad \texttt{writeElement(dataFILE)} \,; \nonumber
\end{equation}
where \texttt{dataFILE} is the a file instance managed by \texttt{BeamIO}.

\paragraph{Utilities} ~\newline
\noindent
The \texttt{FocusQuadrupole} derived class has no utilities.

\FloatBarrier

\subsubsection{Derived class: \texttt{DefocusQuadrupole(BeamLineElement)}}

\paragraph{Instantiation} ~\newline
\noindent
The call to instantiate the \texttt{DefocusQuadrupole} derived class is:
\begin{center}
  \texttt{DefocusQuadrupole(Name, rStrt, vStrt, drStrt, dvStrt, Length,
          Strength, kDQ)} 
\end{center}
Parent class arguments \texttt{Name}, \texttt{rStrt}, \texttt{vStrt},
\texttt{drStrt}, and \texttt{dvStrt} are described in
section~\ref{SubSubSect:BLE:InstAttr}.
These arguments are passed directly to \texttt{BeamLineElement}.
The \texttt{DefocusQuadrupole} arguments are translated into instance
attributes as described in section~\ref{SubSubSect:DQuad:InstAttr} and
defined in table~\ref{Tab:DQuad:Attributes}.
The quadrupole \texttt{Length} is required together with
either the field gradient, \texttt{Strength}
(equation~\ref{Eq:Trnsf:gxy}), or the quadrupole $k$
parameter, \texttt{kDQ} (equation~\ref{Eq:Effectivekq}).

\paragraph{Instance attributes and access methods} ~\newline
\label{SubSubSect:DQuad:InstAttr}
\noindent
The instance attributes are defined in
table~\ref{Tab:DQuad:Attributes}. 
The attributes are accessed and set using the methods defined in
table~\ref{Tab:DQuad:Methods}.
\begin{table}[h]
  \caption{
    Definition of attributes of instances of
    the \texttt{DefocusQuadrupole(BeamLineElement)} derived class.
    All attributes are required in the call to instantiate the
    element.
  }
  \label{Tab:DQuad:Attributes}
  \begin{center}
    \begin{tabular}{|l|c|c|p{10cm}|}
      \hline
      \textbf{Attribute}   & \textbf{Type} & \textbf{Unit} & \textbf{Comment}                    \\
      \hline
      \texttt{DQmode}   & int   &       & If 0, use particle momentum in calculation of transfer
                                          matrix; if 1, use reference particle momentum.         \\
      \texttt{Length}   & float & m     & Effective length of quadrupole.                        \\
      \texttt{Strength} & float & T/m   & Magnetic field gradient; required if kDQ is not given. \\
      \texttt{kDQ}      & float & m$^2$ & Quadrupole $k$ paramter.                               \\
      \hline
    \end{tabular}
  \end{center}
\end{table}
\begin{table}[h]
  \caption{
    Definition of access methods for the \texttt{DefocusQuadrupole} derived
    class. 
  }
  \label{Tab:DQuad:Methods}
  \begin{center}
    \begin{tabular}{|c|c|p{7cm}|}
      \hline
      \textbf{Set method} & \textbf{Get method}  & \textbf{Comment}                                    \\
      \hline
      \texttt{setDQmode(DQmode)}   & \texttt{getDQmode()}   & Set/get DQmode.                        \\
      \texttt{setLength(Length)}   & \texttt{getLength()}   & Set/get length.                        \\
      \texttt{setStrength(Length)} & \texttt{getStrength()} & Set/get strength (field gradient).     \\
      \texttt{setKDQ(Length)}      & \texttt{getKDQ()}      & Set/get kDQ, quadrupole $k$ parameter. \\
      \texttt{setTransferMatrix()} &                        & Set transfer matrix.                   \\
      \hline
    \end{tabular}
  \end{center}
\end{table}

\paragraph{Processing methods} ~\newline
\noindent
The \texttt{DefocusQuadrupole} processing method is defined in
table~\ref{Tab:DQuad:Methods}.
\begin{table}[h]
  \caption{
    Utilities provided by the \texttt{DefocusQuadrupole} derived
    class. 
  }
  \label{Tab:DQuad:Methods}
  \begin{center}
    \begin{tabular}{|l|c|c|p{7cm}|}
      \hline
      \textbf{Method} & \textbf{Argument(s)} & \textbf{Return} & \textbf{Comment}                     \\
      \hline
      \texttt{calckDQ()} &  & float & Calculates \texttt{kDQ} if strength is specified.               \\
      \texttt{calcStrength()} &  & float & Calculates \texttt{Strength} if \texttt{kDQ} is specified. \\
      \hline
    \end{tabular}
  \end{center}
\end{table}

\paragraph{I/o methods} ~\newline
\noindent
Methods to read and write instance attributes to the files defined
using the \texttt{BeamIO} package (see section \ref{Sect:BeamIO} are
provided.
The calls are:
\begin{equation}
  \texttt{readElement(dataFILE)} \quad \quad {\rm and }
      \quad \quad \texttt{writeElement(dataFILE)} \,; \nonumber
\end{equation}
where \texttt{dataFILE} is the a file instance managed by \texttt{BeamIO}.

\paragraph{Utilities} ~\newline
\noindent
The \texttt{DefocusQuadrupole} derived class has no utilities.

\FloatBarrier

\subsubsection{Derived class: \texttt{SectorDipole(BeamLineElement)}}

\paragraph{Instantiation} ~\newline
\noindent
The call to instantiate the \texttt{SectorDipole} derived class is:
\begin{center}
  \texttt{SectorDipole(Name, rStrt, vStrt, drStrt, dvStrt, 
          Angle, B)} 
\end{center}
Parent class arguments \texttt{Name}, \texttt{rStrt}, \texttt{vStrt},
\texttt{drStrt}, and \texttt{dvStrt} are described in
section~\ref{SubSubSect:BLE:InstAttr}.
These arguments are passed directly to \texttt{BeamLineElement}.
The \texttt{SectorDipole} arguments are translated into instance
attributes as described in section~\ref{SubSubSect:SDpl:InstAttr} and
defined in table~\ref{Tab:SDpl:Attributes}.

The orientation of the RLPC coordinate axes with respect to those of
the laboratory frame changes from the start of sector dipole to its
end.
Referring to figure~\ref{fig:Dipole}, the vector, $\bm{v}_{\rm ES}$,
that translates the origin of the RLPC coordinate system at the start
of the sector dipole to the origin of the RLPC coordinate system at
its end is given by:
\begin{equation}
  \bm{v}_{\rm ES} = 2 \rho_0 \sin\left( \frac{\phi}{2} \right)
                     \begin{pmatrix}
                       \sin\left( \frac{\phi}{2} \right) \\
                       0                                 \\
                       \cos\left( \frac{\phi}{2} \right)
                     \end{pmatrix} \,;
\end{equation}
where $\rho_0$ is the radius of the circular locus of the trajectory
of the reference particle.
If the rotation matrix taking the RPLC axes at the start of the sector
dipole to the laboratory coordinate axes is
$\underline{\underline{R}}_{\rm S}$, then the vector,
$\bm{v}^{\rm lab}_{\rm ES}$, that translates from the start 
of the sector dipole to its end in laboratory coordinates is given by:
\begin{equation}
  \bm{v}^{\rm lab}_{\rm ES} = \underline{\underline{R}}_{\rm S} \bm{v}_{\rm ES} \,.
\end{equation}
The rotation matrix that transforms from the RPLC system at the end
of the sector dipole to the laboratory coordinate system,
$\underline{\underline{R}}_{\rm E}$ is given by:
\begin{equation}
  \underline{\underline{R}}_{\rm E} =
      \underline{\underline{R}}_{\rm S} \underline{\underline{R}} \, ;
\end{equation}
where:
\begin{equation}
  \underline{\underline{R}} = 
        \begin{pmatrix}
          \cos \phi & 0 & -\sin \phi \\
          0         & 1 &  0         \\
          \sin \phi & 0 &  \cos \phi 
        \end{pmatrix} \, .
\end{equation}

\paragraph{Instance attributes and access methods} ~\newline
\label{SubSubSect:SDpl:InstAttr}
\noindent
The instance attributes are defined in
table~\ref{Tab:SDpl:Attributes}. 
The attributes are accessed and set using the methods defined in
table~\ref{Tab:SDpl:Methods}.
\begin{table}[h]
  \caption{
    Definition of attributes of instances of
    the \texttt{SectorDipole(BeamLineElement)} derived class.
    All attributes are required in the call to instantiate the
    element.
  }
  \label{Tab:SDpl:Attributes}
  \begin{center}
    \begin{tabular}{|l|c|c|p{10cm}|}
      \hline
      \textbf{Attribute}   & \textbf{Type} & \textbf{Unit} & \textbf{Comment}                      \\
      \hline
      \texttt{Angle}   & float & rad & Angle through which sector dipole bends positive reference
                                 particle.                                                         \\
      \texttt{B}       & float & T   & Magnetic field.                                             \\
      \hline
    \end{tabular}
  \end{center}
\end{table}
\begin{table}[h]
  \caption{
    Definition of access methods for the \texttt{SectorDipole} derived
    class. 
  }
  \label{Tab:SDpl:Methods}
  \begin{center}
    \begin{tabular}{|c|c|p{7cm}|}
      \hline
      \textbf{Set method} & \textbf{Get method}  & \textbf{Comment}       \\
      \hline
      \texttt{setAngle(Angle)}     & \texttt{getAngle()}  & Set/get bending angle.         \\
      \texttt{setB(B)}             & \texttt{getB()}      & Set/get dipole magnetic field. \\
      \texttt{setLength()}         & \texttt{getLength()} & Set/get length of reference
                                                            particle trajectory through
                                                            sector dipole (arc length).    \\
      \texttt{setTransferMatrix()} &                      & Set transfer matrix.           \\
      \hline
    \end{tabular}
  \end{center}
\end{table}

\paragraph{Processing methods} ~\newline
\noindent
The \texttt{SectorDipole} derived class has no processing methods.

\paragraph{I/o methods} ~\newline
\noindent
Methods to read and write instance attributes to the files defined
using the \texttt{BeamIO} package (see section \ref{Sect:BeamIO} are
provided.
The calls are:
\begin{equation}
  \texttt{readElement(dataFILE)} \quad \quad {\rm and }
      \quad \quad \texttt{writeElement(dataFILE)} \,; \nonumber
\end{equation}
where \texttt{dataFILE} is the a file instance managed by \texttt{BeamIO}.

\paragraph{Utilities} ~\newline
\noindent
The \texttt{SectorDipole} derived class has no utilities.

\FloatBarrier

\subsubsection{Derived class: \texttt{Solenoid(BeamLineElement)}}

\paragraph{Instantiation} ~\newline
\noindent
The call to instantiate the \texttt{Solenoid} derived class is:
\begin{center}
  \texttt{Solenoid(Name, rStrt, vStrt, drStrt, dvStrt,
          Length, Strength, kSol)} 
\end{center}
Parent class arguments \texttt{Name}, \texttt{rStrt}, \texttt{vStrt},
\texttt{drStrt}, and \texttt{dvStrt} are described in
section~\ref{SubSubSect:BLE:InstAttr}.
These arguments are passed directly to \texttt{BeamLineElement}.
The \texttt{Solenoid} arguments are translated into instance
attributes as described in section~\ref{SubSubSect:Slnd:InstAttr} and
defined in table~\ref{Tab:Slnd:Attributes}.
The solenoid \texttt{Length} is required together with either the
magnetic field strength, \texttt{Strength} or the solenoid $k$
parameter, \texttt{kSol} (equation~\ref{Eq:Effectiveks}).  

\paragraph{Instance attributes and access methods} ~\newline
\label{SubSubSect:Slnd:InstAttr}
\noindent
The instance attributes are defined in
table~\ref{Tab:Slnd:Attributes}. 
The attributes are accessed and set using the methods defined in
table~\ref{Tab:Slnd:Methods}.
\begin{table}[h]
  \caption{
    Definition of attributes of instances of
    the \texttt{Solenoid(BeamLineElement)} derived class.
    All attributes are required in the call to instantiate the
    element.
  }
  \label{Tab:Slnd:Attributes}
  \begin{center}
    \begin{tabular}{|l|c|c|p{10cm}|}
      \hline
      \textbf{Attribute}   & \textbf{Type} & \textbf{Unit} & \textbf{Comment}                     \\
      \hline
      \texttt{Length}   & float & m     & Effective length of solenoid.                           \\
      \texttt{Strength} & float & T/m   & Magnetic field gradient; required if kSol is not given. \\
      \texttt{kSol}     & float & m$^2$ & Solenoid $k$ paramter.                                  \\
      \hline
    \end{tabular}
  \end{center}
\end{table}
\begin{table}[h]
  \caption{
    Definition of access methods for the \texttt{Solenoid} derived
    class. 
  }
  \label{Tab:Slnd:Methods}
  \begin{center}
    \begin{tabular}{|c|c|p{7cm}|}
      \hline
      \textbf{Set method} & \textbf{Get method}  & \textbf{Comment}                                  \\
      \hline
      \texttt{setLength(Length)}   & \texttt{getLength()}   & Set/get length.                        \\
      \texttt{setStrength(Length)} & \texttt{getStrength()} & Set/get strength (field gradient).     \\
      \texttt{setKFQ(Length)}      & \texttt{getKFQ()}      & Set/get kSol, solenoid $k$ parameter. \\
      \texttt{setTransferMatrix()} &                        & Set transfer matrix.                   \\
      \hline
    \end{tabular}
  \end{center}
\end{table}

\paragraph{Processing methods} ~\newline
\noindent
The \texttt{Solenoid} processing method is defined in
table~\ref{Tab:Slnd:Methods}.
\begin{table}[h]
  \caption{
    Utilities provided by the \texttt{Solenoid} derived
    class. 
  }
  \label{Tab:Slnd:Methods}
  \begin{center}
    \begin{tabular}{|l|c|c|p{7cm}|}
      \hline
      \textbf{Method} & \textbf{Argument(s)} & \textbf{Return} & \textbf{Comment}                      \\
      \hline
      \texttt{calckSol()}     &  & float & Calculates \texttt{kSol} if strength is specified.          \\
      \texttt{calcStrength()} &  & float & Calculates \texttt{Strength} if \texttt{kSol} is specified. \\
      \hline
    \end{tabular}
  \end{center}
\end{table}

\paragraph{I/o methods} ~\newline
\noindent
Methods to read and write instance attributes to the files defined
using the \texttt{BeamIO} package (see section \ref{Sect:BeamIO} are
provided.
The calls are:
\begin{equation}
  \texttt{readElement(dataFILE)} \quad \quad {\rm and }
      \quad \quad \texttt{writeElement(dataFILE)} \,; \nonumber
\end{equation}
where \texttt{dataFILE} is the a file instance managed by \texttt{BeamIO}.

\paragraph{Utilities} ~\newline
\noindent
The \texttt{Solenoid} derived class has no utilities.
