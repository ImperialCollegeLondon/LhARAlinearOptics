\graphicspath{ {06-MCD/Figures/} }

\section{Module, class and data structures}

The linear optics package has been written in object-oriented Python
and is broken down in four principal modules:
\begin{itemize}
  \item \underline{\texttt{BeamLineElement}:} provides descriptions
    of the various beam-line elements required to build a description
    of the beam line.
    Each individual element, such as a drift, quadrupole, etc., is
    described in a class derived from the \texttt{BeamLineElement}
    parent class.
  \item \underline{\texttt{BeamLine}:} provides code to assemble the
    elements into a coherent beam line.
    \texttt{BeamLine} is a singleton class to ensure that two beam
    lines can not be simulated in a single run of the package.
  \item \underline{\texttt{Beam}:} provides code to calculate
    ensemble properties of the beam such as emittance, etc.
    The ensemble properties are stored as instance attributes of
    the \texttt{Beam} class.
  \item \underline{\texttt{Particle}:} provides code to record beam
    particles at positions along the beam line.
    The module provides the singleton \texttt{ReferenceParticle}
    class derived from the \texttt{Particle} class.
\end{itemize}
Other modules: \texttt{BeamIO}, \texttt{LaTeX},
\texttt{PhysicalConstants}, \texttt{Report}, \texttt{Simulation}
and \texttt{Utilities} support the principal modules or provide
services.
The data structure is implemented as attributes of the instances of the 
various classes.
This section describes the implementation of the various modules, the
classes of which they are composed, and how access to the data is
provided.

Each class has methods by which to access a list of the class
instances and a boolean flag by which to generate debug print out (see
table~\ref{Tab:ClassAttributeAccess}).
\begin{table}[h]
  \caption{Methods by which to set and access class attributes.}
  \label{Tab:ClassAttributeAccess}
  \begin{center}
    \begin{tabular}{|l|c|l|p{7cm}|}
      \hline
      \textbf{Method}          & \textbf{Argument} & \textbf{Return}            & \textbf{Comment}                      \\
      \hline
      \texttt{getinstances()}  &                   & List of instances of class &                                       \\
      \texttt{setDebug(Debug)} & Boolean           &                            & Sets flag to generate debug print-out \\
      \texttt{getDebug()}      &                   & Boolean debug flag         & If true, generate debug print-out     \\
      \texttt{setAll2None()}   &                   &                            & Set all instance attributes to \texttt{None} at start of instanciation. \\
      \hline
    \end{tabular}
  \end{center}
\end{table}

\subsection{\texttt{BeamLineElement}}

\subsubsection{Parent class}

\paragraph{Instanciation} ~\newline
\noindent
The call to instanciate the \texttt{BeamLineElenent} class is:
\begin{center}
  \texttt{BeamLineElement(Name, rStrt, vStrt, drStrt)}
\end{center}
The arguments are translated into instance attributes as described in
section~\ref{SubSubSect:InstAttr} and defined in
table~\ref{Tab:BLE:Attributes}.

\paragraph{Instance attributes and access methods} ~\newline
\label{SubSubSect:InstAttr}
\noindent
Properties common to all beam-line elements are stored as instance
attributes of the parent \texttt{BeamLineElement} class.
The instance attributes are defined in table~\ref{Tab:BLE:Attributes}.
The attributes are accessed and set using the methods defined in
table~\ref{Tab:BLE:Methods}.
\begin{table}[h]
  \caption{
    Definition of attributes of instances of
    the \texttt{BeamLineElement} class.
    The attributes marked $^*$ above the dividing line are required in
    the call to instanciate the element.
    The attributes marked $^\dagger$ below the dividing line are
    calculated.
  }
  \label{Tab:BLE:Attributes}
  \begin{center}
    \begin{tabular}{|l|c|c|p{10cm}|}
      \hline
      \textbf{Attribute} & \textbf{Type} & \textbf{Unit} & \textbf{Comment}                                                                   \\
      \hline
      \texttt{Name}$^*$     & String        &     & Name of beam-line element.                                                                \\
      \texttt{rStrt}$^*$    & numpy.ndarray & m   & $[x, y, z]$ position of entrance to element in laboratory coordinate system.              \\
      \texttt{vStrt}$^*$    & numpy.ndarray & rad & $[[i],[\theta, \phi]]$ (polar and azimutal angles) of RPLC $y$ and $z$ axes ($i=0,1$ respectively) at start. \\
      \texttt{drStrt}$^*$   & numpy.ndarray & m   & ``Error'', $[x, y, z]$, displacement of start from nominal position (not yet implemented).\\
      \texttt{dvStrt}$^*$   & numpy.ndarray & rad & ``Error'', $[[i],[\theta, \phi]]$, deviation in $\theta$ and $\phi$ from nominal axis (not yet implemented). \\
      \hline
      \texttt{Strt2End}$^\dagger$     & numpy.ndarray &     & $1\times3$ translation from start of element to end; in laboratory coordinates.  Set in derived class. \\
      \texttt{Rot2LbStrt}$^\dagger$   & numpy.ndarray &     & $3\times3$ rotation matrix that takes RPLC axes to laboratory axes at start.     \\
      \texttt{Rot2LbEnd}$^\dagger$    & numpy.ndarray &     & $3\times3$ rotation matrix that takes RPLC axes to laboratory axes at end.  Set in derived class.      \\
      \texttt{TnrsMtrx}$^\dagger$     & numpy.ndarray &     & $3\times3$ transfer matrix.  Set in derived class.                               \\
      \hline
    \end{tabular}
  \end{center}
\end{table}
\begin{table}[h]
  \caption{
    Definition of access methods for the \texttt{BeamLineElement}
    class. 
  }
  \label{Tab:BLE:Methods}
  \begin{center}
    \begin{tabular}{|c|c|p{7cm}|}
      \hline
      \textbf{Set method} & \textbf{Get method}  & \textbf{Comment}                                                         \\
      \hline
      \texttt{setName(Name)}     & \texttt{getName()}       & Set/get name of beam-line element.                                \\
      \texttt{setrStrt(rStrt)}   & \texttt{getrStrt()}      & Set/get laboratory $[x, y, z]$ position of entrance.              \\
                                         \texttt{setvStrt(vStrt)}   & \texttt{getvStrt()}      & Set/get RPLC $[\theta, \phi]$ of principal axis.                  \\
      \texttt{setdrStrt(drStrt)} & \texttt{getdrStrt()}     & Set/get ``error'' displacement.                                   \\
      \texttt{setdvStrt(dvStrt)} & \texttt{getdvStrt()}     & Set/get ``error'' deviation in $[\theta, \phi]$.                  \\
      \texttt{setRot2LbStrt()}   & \texttt{getRot2LbStrt()} & Set/get rotation matrix from RPLC axes to laboratory.             \\
                                 & \texttt{getStrt2End()}   & Get displacement vector start to end in laboratory coordinates.   \\
                                & \texttt{getRot2LabStrt()} & Get rotation matrix from RPLC to laboratory at start.             \\
                                 & \texttt{getRot2LanEnd()} & Get rotation matrix from RPLC to laboratory at end.               \\
                             & \texttt{getTransferMatrix()} & Get transfer matrix set in derived class.                         \\
      \hline
    \end{tabular}
  \end{center}
\end{table}

\paragraph{Processing methods} ~\newline
\noindent
Table~\ref{Tab:BLE:ProcMethods} presents the processing methods provided
in the \texttt{BeamLineElement} class.
\begin{table}[h]
  \caption{
    Processing methodes provided by the \texttt{BeamLineElement}
    class. 
  }
  \label{Tab:BLE:ProcMethods}
  \begin{center}
    \begin{tabular}{|l|c|c|p{7cm}|}
      \hline
      \textbf{Method} & \textbf{Argument(s)} & \textbf{Return} & \textbf{Comment}                                            \\
      \hline
      \texttt{OutsideBeamPipe(R)} & Float                 & Boolean               & Returns False if particle is inside beam pipe.
                                                                                    If \texttt{R}, radial distance from $z$ axis in RPLC,
                                                                                    falls outside beam pipe, returns True. \\
      \texttt{Transport(V)}       & $6\times1$ np.ndarray & $6\times1$ np.ndarray & Transport 6D trace-space vector, \texttt{V}, accross
                                                                                    element.
                                                                                    Final trace-space vector returned.       \\
      \texttt{Shit2Local(V)}      & $6\times1$ np.ndarray & $6\times1$ np.ndarray & Transform 6D trace-space vector, \texttt{V}, from RPLC
                                                                                    to laboratory coordinates.
                                                                                    Phase-space vector in laboratory frame returned. \\
      \texttt{Shit2Laboratory(U)} & $6\times1$ np.ndarray & $6\times1$ np.ndarray & Transform 6D phase-space vector, \texttt{U}, from 
                                                                                    laboratory coordinates to trace-space coordaintes in
                                                                                    the RPLC frame.
                                                                                    Trace-space vector in RLPC frame returned. \\
      \hline
    \end{tabular}
  \end{center}
\end{table}

\paragraph{I/o methods} ~\newline
\noindent
Methods to read and write instance attributes to the files defined
using the \texttt{BeamIO} package (see section \ref{Sect:BeamIO} are
provided.
The calls are:
\begin{equation}
  \texttt{readElement(dataFILE)} \quad \quad {\rm and }
      \quad \quad \texttt{writeElement(dataFILE)} \,; \nonumber
\end{equation}
where \texttt{dataFILE} is the a file instance managed by \texttt{BeamIO}.

\paragraph{Utilities} ~\newline
\noindent
Table~\ref{Tab:BLE:Utilities} presents the utilities provided in the
\texttt{BeamLineElement} class.
\begin{table}[h]
  \caption{
    Utilities provided by the \texttt{BeamLineElement}
    class. 
  }
  \label{Tab:BLE:Utilities}
  \begin{center}
    \begin{tabular}{|l|c|c|p{7cm}|}
      \hline
      \textbf{Method} & \textbf{Argument(s)} & \textbf{Return} & \textbf{Comment}                                            \\
      \hline
      \texttt{cleaninstances()}     &                 &  & Delete (using ``del'') all instances of the \texttt{BeamLineElement} class.
                                                           Reset \texttt{instances} list. \\
      \texttt{removeInstance(inst)} & Instance of BLE &  & Remove instance \texttt{inst} and remove from list of instances of
                                                           \texttt{BeamLineElement}. \\
      \hline
    \end{tabular}
  \end{center}
\end{table}
