\graphicspath{ {06-MCD/Figures/} }

\section{Module, class and data structures}

The linear optics package has been written in object-oriented Python
and is broken down in four principal modules:
\begin{itemize}
  \item \underline{\texttt{BeamLineElement}:} provides the various
    beam-line elements required to build a description of the beam
    line.
    Each individual element, such as a drift, quadrupole, etc., is
    described in a class derived from the \texttt{BeamLineElement}
    parent class.
  \item \underline{\texttt{BeamLine}:} provides code to assemble the
    elements into a coherent beam line.
    \texttt{BeamLine} is a singleton class to ensure that two beam
    lines can not be simulated in a single run of the package.
    The \texttt{extrapolateBeam} class is derived from
    the \texttt{Beam} class to handle the propagation of beam
    envelopes without the need to track individual particles.
  \item \underline{\texttt{Beam}:} provides code to calculate
    ensemble properties of the beam such as emittance.
    The ensemble properties are stored as instance attributes of
    the \texttt{Beam} class.
  \item \underline{\texttt{Particle}:} provides code to record beam
    particles at positions along the beam line.
    The module provides the singleton \texttt{ReferenceParticle}
    class derived from the \texttt{Particle} class.
\end{itemize}
Other modules: \texttt{BeamIO}, \texttt{LaTeX},
\texttt{PhysicalConstants}, \texttt{Report}, \texttt{Simulation},
\texttt{UserFramework}, \texttt{Utilities}, and \texttt{Visualise}
support the principal modules or provide services.
The data structure is implemented as attributes of the instances of the 
various classes.
This section describes the implementation of the various modules, the
classes of which they are composed, and how access to the data is
provided.

Each class has methods by which to access a list of the class
instances and a Boolean flag by which to generate debug print out (see
table~\ref{Tab:ClassAttributeAccess}).
\begin{table}[h]
  \caption{Methods by which to set and access class attributes.}
  \label{Tab:ClassAttributeAccess}
  \begin{center}
    \begin{tabular}{|l|c|l|p{7cm}|}
      \hline
      \textbf{Method}          & \textbf{Argument} & \textbf{Return}            & \textbf{Comment}                      \\
      \hline
      \texttt{getinstances()}  &                   & List of instances of class &                                       \\
      \texttt{setDebug(Debug)} & Boolean           &                            & Sets flag to generate debug print-out \\
      \texttt{getDebug()}      &                   & Boolean debug flag         & If True, generate debug print-out     \\
      \texttt{setAll2None()}   &                   &                            & Set all instance attributes to
                                                                                  \texttt{None} at start of instantiation. \\
      \texttt{SummaryStr()}    &                   & String                     & Text sring to record paramters in debug
                                                                                  print out. \\
      \hline
    \end{tabular}
  \end{center}
\end{table}

\subsection{\texttt{BeamLineElement}}
\label{SubSubSect:BLE}

\subsubsection{Parent class}

\paragraph{Instantiation}   ~\newline
\noindent
The call to instantiate the \texttt{BeamLineElenent} class is:
\begin{center}
  \texttt{BeamLineElement(Name, rStrt, vStrt, drStrt, dvStrt)}
\end{center}
where:
\begin{description}
  \item\texttt{Name}: (string) is the unique name of the element;
  \item\texttt{rStrt}: (numpy.ndarray(3)) is the three-vector
    position in laboratory coordinates of the start of the element;
  \item\texttt{vStrt}: (numpy.ndarray(1,2)) is the polar, $\theta$,
    and azimuthal, $\phi$, angles that define the $y$ ($i=0$) and $z$
    ($i=1$)  axes of the RPLC coordinate system at the start of the
    element ($\texttt{vStrt}=[[i],[\theta, \phi]]$);  
  \item\texttt{drStrt}: (numpy.ndarray(3)) error in the three-vector
    position with respect to the nominal position; and
  \item\texttt{dvStrt}: (numpy.ndarray(1,2)) error in the polar and
    azimuthal angles defining RLPC the $y$ and $z$ axes.
\end{description}
All arguments are required.

\paragraph{Instance attributes and access methods} ~\newline
\label{SubSubSect:BLE:InstAttr}
\noindent
Properties common to all beam-line elements are stored as instance
attributes of the parent \texttt{BeamLineElement} class.
The instance attributes are defined in table~\ref{Tab:BLE:Attributes}.
The attributes are accessed and set using the methods defined in
table~\ref{Tab:BLE:Methods}.
\begin{table}[h]
  \caption{
    Definition of attributes of instances of
    the \texttt{BeamLineElement} class.
    The attributes marked $^*$ above the dividing line are required in
    the call to instantiate the element.
    The attributes marked $^\dagger$ below the dividing line are
    calculated.
  }
  \label{Tab:BLE:Attributes}
  \begin{center}
    \begin{tabular}{|l|c|c|p{9.7cm}|}
      \hline
      \textbf{Attribute} & \textbf{Type} & \textbf{Unit} & \textbf{Comment}                                                                   \\
      \hline
      \texttt{Name}$^*$     & String        &     & Name of beam-line element.                                                                \\
      \texttt{rStrt}$^*$    & numpy.ndarray & m   & $[x, y, z]$ position of entrance to element in laboratory coordinate system.              \\
      \texttt{vStrt}$^*$    & numpy.ndarray & rad & $[[i],[\theta, \phi]]$ (polar and azimuthal angles) of RPLC $y$ and $z$ axes ($i=0,1$ respectively) at start. \\
      \texttt{drStrt}$^*$   & numpy.ndarray & m   & ``Error'', $[x, y, z]$, displacement of start from nominal position (not yet implemented).\\
      \texttt{dvStrt}$^*$   & numpy.ndarray & rad & ``Error'', $[[i],[\theta, \phi]]$, deviation in $\theta$ and $\phi$ from nominal axis (not yet implemented). \\
      \hline
      \texttt{Strt2End}$^\dagger$     & numpy.ndarray &     & $1\times3$ translation from start of element to end; in laboratory coordinates.  Set in derived class. \\
      \texttt{Rot2LbStrt}$^\dagger$   & numpy.ndarray &     & $3\times3$ rotation matrix that takes RPLC axes to laboratory axes at start.     \\
      \texttt{Rot2LbEnd}$^\dagger$    & numpy.ndarray &     & $3\times3$ rotation matrix that takes RPLC axes to laboratory axes at end.  Set in derived class.      \\
      \texttt{TnrsMtrx}$^\dagger$     & numpy.ndarray &     & $3\times3$ transfer matrix.  Set in derived class.                               \\
      \hline
    \end{tabular}
  \end{center}
\end{table}
\begin{sidewaystable}[h]
  \caption{
    Definition of access methods for the \texttt{BeamLineElement}
    class. 
  }
  \label{Tab:BLE:Methods}
  \begin{center}
    \begin{tabular}{|c|c|p{10cm}|}
      \hline
      \textbf{Set method} & \textbf{Get method}  & \textbf{Comment}                                                         \\
      \hline
      \texttt{setName(Name)}     & \texttt{getName()}        & Set/get name of beam-line element.                                \\
      \texttt{setrStrt(rStrt)}   & \texttt{getrStrt()}       & Set/get laboratory $[x, y, z]$ position of entrance.              \\
      \texttt{setvStrt(vStrt)}   & \texttt{getvStrt()}       & Set/get RPLC $[\theta, \phi]$ of principal axis at start of element. \\
                                 & \texttt{getvEnd()}        & Set/get RPLC $[\theta, \phi]$ of principal axis at end of element.   \\
      \texttt{setdrStrt(drStrt)} & \texttt{getdrStrt()}      & Set/get ``error'' displacement.                                   \\
      \texttt{setdvStrt(dvStrt)} & \texttt{getdvStrt()}      & Set/get ``error'' deviation in $[\theta, \phi]$.                  \\
      \texttt{setLength(length)} & \texttt{getLength}        & Set/get increment in $s$ across element, (length for elements that do not bend beam). \\
      \texttt{setRot2LbStrt()}   & \texttt{getRot2LbStrt()}  & Set/get rotation matrix from RPLC axes to laboratory.             \\
      \texttt{setRot2LabStrt()}  & \texttt{getRot2LbStrt()}  & Setget rotation matrix from RPLC to laboratory at start.             \\
      \texttt{setStrt2End(t)}    & \texttt{getStrt2End()}    & Set/get displacement vector start to end in laboratory coordinates.
                                                               setStrt2End takes 1 argument, \texttt{t}, a 1D np.ndarray containing
                                                               the translation from the start to the end of the element in RPLC. \\
      \texttt{setRot2LbEnd(R)}  & \texttt{getRot2LbEnd()}    & Set/get rotation matrix from RPLC to laboratory at end.
                                                               setRot2LbEnd takes 1 argument, \texttt{R}, a 2D np.array containing
                                                               the rotation matrix to be set.                                    \\
                                & \texttt{getTransferMatrix()}  & Get transfer matrix set in derived class.                         \\
                                & \texttt{getLines()}        & Get lines to write LaTeX specification of element.                   \\
      \hline
    \end{tabular}
  \end{center}
\end{sidewaystable}

\paragraph{Processing methods} ~\newline
\noindent
Table~\ref{Tab:BLE:ProcMethods} presents the processing methods provided
in the \texttt{BeamLineElement} class.
\begin{sidewaystable}[h]
  \caption{
    Processing methods provided by the \texttt{BeamLineElement}
    class. 
  }
  \label{Tab:BLE:ProcMethods}
  \begin{center}
    \begin{tabular}{|l|c|c|p{12cm}|}
      \hline
      \textbf{Method} & \textbf{Argument(s)} & \textbf{Return} & \textbf{Comment}                                            \\
      \hline
      \texttt{OutsideBeamPipe(R)} & Float                 & Boolean               & Returns False if particle is inside beam pipe.
                                                                                    If \texttt{R}, radial distance from $z$ axis in RPLC,
                                                                                    falls outside beam pipe, returns True. \\
      \texttt{ExpansionParamterFail(R)} & Float           & Boolean               & Calculates an approximate expansion parameter and
                                                                                    returns False if the parameter is large ($>1$).
                                                                                    Not yet used in \texttt{Transport}.      \\
      \texttt{Transport(V)}       & $6\times1$ np.ndarray & $6\times1$ np.ndarray & Transport 6D trace-space vector, \texttt{V}, across
                                                                                    element.
                                                                                    Final trace-space vector returned.       \\
      \texttt{Shit2Local(V)}      & $6\times1$ np.ndarray & $6\times1$ np.ndarray & Transform 6D trace-space vector, \texttt{V}, from RPLC
                                                                                    to laboratory coordinates.
                                                                                    Phase-space vector in laboratory frame returned. \\
      \texttt{Shit2Laboratory(U)} & $6\times1$ np.ndarray & $6\times1$ np.ndarray & Transform 6D phase-space vector, \texttt{U}, from 
                                                                                    laboratory coordinates to trace-space coordinates in
                                                                                    the RPLC frame.
                                                                                    Trace-space vector in RLPC frame returned. \\
      \hline
    \end{tabular}
  \end{center}
\end{sidewaystable}

\paragraph{I/o methods} ~\newline
\noindent
Methods to read and write instance attributes to the files defined
using the \texttt{BeamIO} package (see section \ref{Sect:BeamIO} are
provided.
The calls are:
\begin{equation}
  \texttt{readElement(dataFILE)} \quad \quad {\rm and }
      \quad \quad \texttt{writeElement(dataFILE)} \,; \nonumber
\end{equation}
where \texttt{dataFILE} is \texttt{BeamIO} instance. 

\paragraph{Utilities} ~\newline
\noindent
Table~\ref{Tab:BLE:Utilities} presents the utilities provided in the
\texttt{BeamLineElement} class.
\begin{sidewaystable}[h]
  \caption{
    Utilities provided by the \texttt{BeamLineElement}
    class. 
  }
  \label{Tab:BLE:Utilities}
  \begin{center}
    \begin{tabular}{|l|c|c|p{10cm}|}
      \hline
      \textbf{Method} & \textbf{Argument(s)} & \textbf{Return} & \textbf{Comment}                                            \\
      \hline
      \texttt{cleaninstances()}     &                 &  & Delete (using ``del'') all instances of the \texttt{BeamLineElement} class.
                                                           Reset \texttt{instances} list. \\
      \texttt{removeInstance(inst)} & Instance of BLE &  & Remove instance \texttt{inst} and remove from list of instances of
                                                           \texttt{BeamLineElement}. \\
      \texttt{visualise(axs, CoordSys,Proj)} & \texttt{axs} -- MatPlotLib ``axes'' instance &  & Manages plotting (visualisation) of element.  \\
                                             & \texttt{CoordSys} -- string                  &  & ``Lab'' or ``RPLC'', coordinate system in which to visualise element. \\
                                             & \texttt{Proj} -- string                      &  & ``$xz$'' or ``$yz$'' projection to visualise. \\
      \hline
    \end{tabular}
  \end{center}
\end{sidewaystable}

\FloatBarrier

\subsubsection{Derived class: \texttt{Facility(BeamLineElement)}}

\paragraph{Instantiation} ~\newline
\noindent
The call to instantiate the \texttt{Facility} derived class is:
\begin{center}
  \texttt{FacilityName, rStrt, vStrt, drStrt, dvStrt, p0, VCMV)}
\end{center}
Parent class arguments \texttt{Name}, \texttt{rStrt}, \texttt{vStrt},
\texttt{drStrt}, and \texttt{dvStrt} are described in
section~\ref{SubSubSect:BLE:InstAttr}.
These arguments are passed directly to \texttt{BeamLineElement}.

\paragraph{Instance attributes and access methods} ~\newline
\label{SubSubSect:Fclty:InstAttr}
\noindent
The instance attributes are defined in
table~\ref{Tab:Fclty:Attributes}. 
The attributes are accessed and set using the methods defined in
table~\ref{Tab:Fclty:Methods}.
\begin{table}[h]
  \caption{
    Definition of attributes of instances of
    the \texttt{Facility(BeamLineElement)} derived class.
    All attributes are required in the call to instantiate the
    element.
  }
  \label{Tab:Fclty:Attributes}
  \begin{center}
    \begin{tabular}{|l|c|c|p{10cm}|}
      \hline
      \textbf{Attribute} & \textbf{Type} & \textbf{Unit} & \textbf{Comment}                                                                   \\
      \hline
      \texttt{p0}   & float & MeV & Kinetic energy of reference particle. \\
      \texttt{VCMV} & float & m   & Radius of vacuum-chamber mother
                                    volume.
                                    The radius defines edge of the volume
                                    at which a particle trajectory is
                                    terminated.
                                    It may be necessary to introduce a beam
                                    pipe later.                            \\
      \hline
    \end{tabular}
  \end{center}
\end{table}
\begin{table}[h]
  \caption{
    Definition of access methods for the \texttt{Facility} derived
    class. 
  }
  \label{Tab:Fclty:Methods}
  \begin{center}
    \begin{tabular}{|c|c|p{7cm}|}
      \hline
      \textbf{Set method} & \textbf{Get method}  & \textbf{Comment}                                   \\
      \hline
      \texttt{setp0(Name)}   & \texttt{getp0()}    & Set/get momentum of reference particle (in MeV). \\
      \texttt{setVCMV(VCMV)} & \texttt{getrVCMV()} & Set/get radius of vacuum chamber mother volume.  \\
      \hline
    \end{tabular}
  \end{center}
\end{table}

\paragraph{Processing methods} ~\newline
\noindent
The \texttt{Facility} derived class has no processing methods.

\paragraph{I/o methods} ~\newline
\noindent
Methods to read and write instance attributes to the files defined
using the \texttt{BeamIO} package (see section \ref{Sect:BeamIO} are
provided.
The calls are:
\begin{equation}
  \texttt{readElement(dataFILE)} \quad \quad {\rm and }
      \quad \quad \texttt{writeElement(dataFILE)} \,; \nonumber
\end{equation}
where \texttt{dataFILE} is the a file instance managed by \texttt{BeamIO}.


\paragraph{Utilities} ~\newline
\noindent
The \texttt{Facility} derived class has no utilities.

\FloatBarrier

\subsubsection{Derived class: \texttt{Drift(BeamLineElement)}}

\paragraph{Instantiation} ~\newline
\noindent
The call to instantiate the \texttt{Drift} derived class is:
\begin{center}
  \texttt{Drift(Name, rStrt, vStrt, drStrt, dvStrt, Length)}
\end{center}
Parent class arguments \texttt{Name}, \texttt{rStrt}, \texttt{vStrt},
\texttt{drStrt}, and \texttt{dvStrt} are described in
section~\ref{SubSubSect:BLE:InstAttr}.
These arguments are passed directly to \texttt{BeamLineElement}.
The \texttt{Drift} arguments are translated into instance
attributes as described in section~\ref{SubSubSect:Drft:InstAttr} and
defined in table~\ref{Tab:Drft:Attributes}.

\paragraph{Instance attributes and access methods} ~\newline
\label{SubSubSect:Drft:InstAttr}
\noindent
The instance attributes are defined in
table~\ref{Tab:Drft:Attributes}. 
The attributes are accessed and set using the methods defined in
table~\ref{Tab:Drft:Methods}.
\begin{table}[h]
  \caption{
    Definition of attributes of instances of
    the \texttt{Drift(BeamLineElement)} derived class.
    All attributes are required in the call to instantiate the
    element.
  }
  \label{Tab:Drft:Attributes}
  \begin{center}
    \begin{tabular}{|l|c|c|p{10cm}|}
      \hline
      \textbf{Attribute} & \textbf{Type} & \textbf{Unit} & \textbf{Comment}                                                                   \\
      \hline
      \texttt{Length} & float & m & Length of drift. \\
      \hline
    \end{tabular}
  \end{center}
\end{table}
\begin{table}[h]
  \caption{
    Definition of access methods for the \texttt{Facility} derived
    class. 
  }
  \label{Tab:Drft:Methods}
  \begin{center}
    \begin{tabular}{|c|c|p{7cm}|}
      \hline
      \textbf{Set method} & \textbf{Get method}  & \textbf{Comment}                       \\
      \hline
      \texttt{setLength(Length)}   & \texttt{getLength()} & Set/get length of drift (in m). \\
      \texttt{setTransferMatrix()} &                      & Set transfer matrix.            \\
      \hline
    \end{tabular}
  \end{center}
\end{table}

\paragraph{Processing methods} ~\newline
\noindent
The \texttt{Facility} derived class has no processing methods.

\paragraph{I/o methods} ~\newline
\noindent
Methods to read and write instance attributes to the files defined
using the \texttt{BeamIO} package (see section \ref{Sect:BeamIO} are
provided.
The calls are:
\begin{equation}
  \texttt{readElement(dataFILE)} \quad \quad {\rm and }
      \quad \quad \texttt{writeElement(dataFILE)} \,; \nonumber
\end{equation}
where \texttt{dataFILE} is the a file instance managed by \texttt{BeamIO}.

\paragraph{I/o methods} ~\newline
\noindent
The \texttt{Facility} derived class has no utilities.

\FloatBarrier

\subsubsection{Derived class: \texttt{Aperture(BeamLineElement)}}

\paragraph{Instantiation} ~\newline
\noindent
The call to instantiate the \texttt{Aperture} derived class is:
\begin{center}
  \texttt{Aperture(Name, rStrt, vStrt, drStrt, dvStrt, ParamList)}
\end{center}
Parent class arguments \texttt{Name}, \texttt{rStrt}, \texttt{vStrt},
\texttt{drStrt}, and \texttt{dvStrt} are described in
section~\ref{SubSubSect:BLE:InstAttr}.
These arguments are passed directly to \texttt{BeamLineElement}.
The \texttt{Aperture} arguments are translated into instance
attributes as described in section~\ref{SubSubSect:Aprtr:InstAttr} and
defined in table~\ref{Tab:Aprtr:Attributes}.

\paragraph{Instance attributes and access methods} ~\newline
\label{SubSubSect:Aprtr:InstAttr}
\noindent
The instance attributes are defined in
table~\ref{Tab:Aprtr:Attributes}. 
The attributes are accessed and set using the methods defined in
table~\ref{Tab:Aprtr:Methods}.
\begin{table}[h]
  \caption{
    Definition of attributes of instances of
    the \texttt{Aperture(BeamLineElement)} derived class.
    All attributes are required in the call to instantiate the
    element.
  }
  \label{Tab:Aprtr:Attributes}
  \begin{center}
    \begin{tabular}{|l|c|c|p{10cm}|}
      \hline
      \textbf{Attribute}   & \textbf{Type} & \textbf{Unit} & \textbf{Comment}                                                                   \\
      \hline
      \texttt{ParamList}   & [] &   & List containing aperture parameters.
                                    The first parameter is an \texttt{int}
                                    and defines the aperture ``\texttt{Type}''.
                                    The remaining elements in the parameter list
                                    are \texttt{float}s with meanings that depend
                                    on \texttt{Type}.                             \\
      \hline
      \texttt{ParamList[0]} & int   &   & \texttt{Type}$=0$; circular                 \\
      \texttt{ParamList[1]} & float & m & Radius of circular aperture                 \\
      \hline
      \texttt{ParamList[0]} & int   &   & \texttt{Type}$=1$; Elliptical                           \\
      \texttt{ParamList[1]} & float & m & Radius of elliptical aperture along $x_{\rm RPLC}$ axis \\
      \texttt{ParamList[2]} & float & m & Radius of elliptical aperture along $y_{\rm RPLC}$ axis \\
      \hline
    \end{tabular}
  \end{center}
\end{table}
\begin{table}[h]
  \caption{
    Definition of access methods for the \texttt{Facility} derived
    class. 
  }
  \label{Tab:Aprtr:Methods}
  \begin{center}
    \begin{tabular}{|c|c|p{7cm}|}
      \hline
      \textbf{Set method} & \textbf{Get method}  & \textbf{Comment}                                  \\
      \hline
      \texttt{setApertureParameters(ParamList)} &                      & Set aperture paramters.
                                                                         Sets \texttt{Type} and parameters
                                                                         depending on \texttt{Type}. \\
                                                & \texttt{getType()} & Get \texttt{Type} of aperture.\\
                                                & \texttt{getParams()} & Get aperture parameters.    \\
      \texttt{setTransferMatrix()}              &                      & Set transfer matrix.        \\
      \hline
    \end{tabular}
  \end{center}
\end{table}

\paragraph{Processing methods} ~\newline
\noindent
The \texttt{Aperture} processing method is defined in
table~\ref{Tab:Aprtr:Methods}.
\begin{table}[h]
  \caption{
    Utilities provided by the \texttt{Aperture} derived
    class. 
  }
  \label{Tab:Aprtr:Methods}
  \begin{center}
    \begin{tabular}{|l|c|c|p{7cm}|}
      \hline
      \textbf{Method} & \textbf{Argument(s)} & \textbf{Return} & \textbf{Comment}                                            \\
      \hline
      \texttt{Transport(V)} & \texttt{np.ndarray} & \texttt{np.ndarray} or \texttt{None} &
                        Transport trace-space vector \texttt{V}.  If \texttt{V} falls outside of the aperture, return \texttt{None}. \\
      \hline
    \end{tabular}
  \end{center}
\end{table}

\paragraph{I/o methods} ~\newline
\noindent
Methods to read and write instance attributes to the files defined
using the \texttt{BeamIO} package (see section \ref{Sect:BeamIO} are
provided.
The calls are:
\begin{equation}
  \texttt{readElement(dataFILE)} \quad \quad {\rm and }
      \quad \quad \texttt{writeElement(dataFILE)} \,; \nonumber
\end{equation}
where \texttt{dataFILE} is the a file instance managed by \texttt{BeamIO}.

\paragraph{Utilities} ~\newline
\noindent
The \texttt{Aperture} derived class has no utilities.

\FloatBarrier

\subsubsection{Derived class: \texttt{FocusQuadrupole(BeamLineElement)}}

\paragraph{Instantiation} ~\newline
\noindent
The call to instantiate the \texttt{FocusQuadrupole} derived class is:
\begin{center}
  \texttt{FocusQuadrupole(Name, rStrt, vStrt, drStrt, dvStrt, Length,
          Strength, kFQ)} 
\end{center}
Parent class arguments \texttt{Name}, \texttt{rStrt}, \texttt{vStrt},
\texttt{drStrt}, and \texttt{dvStrt} are described in
section~\ref{SubSubSect:BLE:InstAttr}.
These arguments are passed directly to \texttt{BeamLineElement}.
The \texttt{FocusQuadrupole} arguments are translated into instance
attributes as described in section~\ref{SubSubSect:FQuad:InstAttr} and
defined in table~\ref{Tab:FQuad:Attributes}.
The quadrupole \texttt{Length} is required together with
either the field gradient, \texttt{Strength}
(equation~\ref{Eq:Trnsf:gxy}), or the quadrupole $k$
parameter, \texttt{kFQ} (equation~\ref{Eq:Effectivekq}).

\paragraph{Instance attributes and access methods} ~\newline
\label{SubSubSect:FQuad:InstAttr}
\noindent
The instance attributes are defined in
table~\ref{Tab:FQuad:Attributes}. 
The attributes are accessed and set using the methods defined in
table~\ref{Tab:FQuad:Methods}.
\begin{table}[h]
  \caption{
    Definition of attributes of instances of
    the \texttt{FocusQuadrupole(BeamLineElement)} derived class.
    All attributes are required in the call to instantiate the
    element.
  }
  \label{Tab:FQuad:Attributes}
  \begin{center}
    \begin{tabular}{|l|c|c|p{10cm}|}
      \hline
      \textbf{Attribute}   & \textbf{Type} & \textbf{Unit} & \textbf{Comment}                    \\
      \hline
      \texttt{FQmode}   & int   &       & If 0, use particle momentum in calculation of transfer
                                          matrix; if 1, use reference particle momentum.         \\
      \texttt{Length}   & float & m     & Effective length of quadrupole.                        \\
      \texttt{Strength} & float & T/m   & Magnetic field gradient; required if kFQ is not given. \\
      \texttt{kFQ}      & float & m$^{-2}$ & Quadrupole $k$ paramter.                               \\
      \hline
    \end{tabular}
  \end{center}
\end{table}
\begin{table}[h]
  \caption{
    Definition of access methods for the \texttt{FocusQuadrupole} derived
    class. 
  }
  \label{Tab:FQuad:Methods}
  \begin{center}
    \begin{tabular}{|c|c|p{7cm}|}
      \hline
      \textbf{Set method} & \textbf{Get method}  & \textbf{Comment}                                    \\
      \hline
      \texttt{setFQmode(FQmode)}   & \texttt{getFQmode()}   & Set/get FQmode.                        \\
      \texttt{setLength(Length)}   & \texttt{getLength()}   & Set/get length.                        \\
      \texttt{setStrength(Length)} & \texttt{getStrength()} & Set/get strength (field gradient).     \\
      \texttt{setKFQ(Length)}      & \texttt{getKFQ()}      & Set/get kFQ, quadrupole $k$ parameter. \\
      \texttt{setTransferMatrix()} &                        & Set transfer matrix.                   \\
      \hline
    \end{tabular}
  \end{center}
\end{table}

\paragraph{Processing methods} ~\newline
\noindent
The \texttt{FocusQuadrupole} processing method is defined in
table~\ref{Tab:FQuad:Methods}.
\begin{table}[h]
  \caption{
    Utilities provided by the \texttt{FocusQuadrupole} derived
    class. 
  }
  \label{Tab:FQuad:Methods}
  \begin{center}
    \begin{tabular}{|l|c|c|p{7cm}|}
      \hline
      \textbf{Method} & \textbf{Argument(s)} & \textbf{Return} & \textbf{Comment}                     \\
      \hline
      \texttt{calckFQ()} &  & float & Calculates \texttt{kFQ} if strength is specified.               \\
      \texttt{calcStrength()} &  & float & Calculates \texttt{Strength} if \texttt{kFQ} is specified. \\
      \hline
    \end{tabular}
  \end{center}
\end{table}

\paragraph{I/o methods} ~\newline
\noindent
Methods to read and write instance attributes to the files defined
using the \texttt{BeamIO} package (see section \ref{Sect:BeamIO} are
provided.
The calls are:
\begin{equation}
  \texttt{readElement(dataFILE)} \quad \quad {\rm and }
      \quad \quad \texttt{writeElement(dataFILE)} \,; \nonumber
\end{equation}
where \texttt{dataFILE} is the a file instance managed by \texttt{BeamIO}.

\paragraph{Utilities} ~\newline
\noindent
The \texttt{FocusQuadrupole} derived class has no utilities.

\FloatBarrier

\subsubsection{Derived class: \texttt{DefocusQuadrupole(BeamLineElement)}}

\paragraph{Instantiation} ~\newline
\noindent
The call to instantiate the \texttt{DefocusQuadrupole} derived class is:
\begin{center}
  \texttt{DefocusQuadrupole(Name, rStrt, vStrt, drStrt, dvStrt, Length,
          Strength, kDQ)} 
\end{center}
Parent class arguments \texttt{Name}, \texttt{rStrt}, \texttt{vStrt},
\texttt{drStrt}, and \texttt{dvStrt} are described in
section~\ref{SubSubSect:BLE:InstAttr}.
These arguments are passed directly to \texttt{BeamLineElement}.
The \texttt{DefocusQuadrupole} arguments are translated into instance
attributes as described in section~\ref{SubSubSect:DQuad:InstAttr} and
defined in table~\ref{Tab:DQuad:Attributes}.
The quadrupole \texttt{Length} is required together with
either the field gradient, \texttt{Strength}
(equation~\ref{Eq:Trnsf:gxy}), or the quadrupole $k$
parameter, \texttt{kDQ} (equation~\ref{Eq:Effectivekq}).

\paragraph{Instance attributes and access methods} ~\newline
\label{SubSubSect:DQuad:InstAttr}
\noindent
The instance attributes are defined in
table~\ref{Tab:DQuad:Attributes}. 
The attributes are accessed and set using the methods defined in
table~\ref{Tab:DQuad:Methods}.
\begin{table}[h]
  \caption{
    Definition of attributes of instances of
    the \texttt{DefocusQuadrupole(BeamLineElement)} derived class.
    All attributes are required in the call to instantiate the
    element.
  }
  \label{Tab:DQuad:Attributes}
  \begin{center}
    \begin{tabular}{|l|c|c|p{10cm}|}
      \hline
      \textbf{Attribute}   & \textbf{Type} & \textbf{Unit} & \textbf{Comment}                    \\
      \hline
      \texttt{DQmode}   & int   &       & If 0, use particle momentum in calculation of transfer
                                          matrix; if 1, use reference particle momentum.         \\
      \texttt{Length}   & float & m     & Effective length of quadrupole.                        \\
      \texttt{Strength} & float & T/m   & Magnetic field gradient; required if kDQ is not given. \\
      \texttt{kDQ}      & float & m$^{-2}$ & Quadrupole $k$ paramter.                               \\
      \hline
    \end{tabular}
  \end{center}
\end{table}
\begin{table}[h]
  \caption{
    Definition of access methods for the \texttt{DefocusQuadrupole} derived
    class. 
  }
  \label{Tab:DQuad:Methods}
  \begin{center}
    \begin{tabular}{|c|c|p{7cm}|}
      \hline
      \textbf{Set method} & \textbf{Get method}  & \textbf{Comment}                                    \\
      \hline
      \texttt{setDQmode(DQmode)}   & \texttt{getDQmode()}   & Set/get DQmode.                        \\
      \texttt{setLength(Length)}   & \texttt{getLength()}   & Set/get length.                        \\
      \texttt{setStrength(Length)} & \texttt{getStrength()} & Set/get strength (field gradient).     \\
      \texttt{setKDQ(Length)}      & \texttt{getKDQ()}      & Set/get kDQ, quadrupole $k$ parameter. \\
      \texttt{setTransferMatrix()} &                        & Set transfer matrix.                   \\
      \hline
    \end{tabular}
  \end{center}
\end{table}

\paragraph{Processing methods} ~\newline
\noindent
The \texttt{DefocusQuadrupole} processing method is defined in
table~\ref{Tab:DQuad:Methods}.
\begin{table}[h]
  \caption{
    Utilities provided by the \texttt{DefocusQuadrupole} derived
    class. 
  }
  \label{Tab:DQuad:Methods}
  \begin{center}
    \begin{tabular}{|l|c|c|p{7cm}|}
      \hline
      \textbf{Method} & \textbf{Argument(s)} & \textbf{Return} & \textbf{Comment}                     \\
      \hline
      \texttt{calckDQ()} &  & float & Calculates \texttt{kDQ} if strength is specified.               \\
      \texttt{calcStrength()} &  & float & Calculates \texttt{Strength} if \texttt{kDQ} is specified. \\
      \hline
    \end{tabular}
  \end{center}
\end{table}

\paragraph{I/o methods} ~\newline
\noindent
Methods to read and write instance attributes to the files defined
using the \texttt{BeamIO} package (see section \ref{Sect:BeamIO} are
provided.
The calls are:
\begin{equation}
  \texttt{readElement(dataFILE)} \quad \quad {\rm and }
      \quad \quad \texttt{writeElement(dataFILE)} \,; \nonumber
\end{equation}
where \texttt{dataFILE} is the a file instance managed by \texttt{BeamIO}.

\paragraph{Utilities} ~\newline
\noindent
The \texttt{DefocusQuadrupole} derived class has no utilities.

\FloatBarrier

\subsubsection{Derived class: \texttt{SectorDipole(BeamLineElement)}}

\paragraph{Instantiation} ~\newline
\noindent
The call to instantiate the \texttt{SectorDipole} derived class is:
\begin{center}
  \texttt{SectorDipole(Name, rStrt, vStrt, drStrt, dvStrt, 
          Angle, B)} 
\end{center}
Parent class arguments \texttt{Name}, \texttt{rStrt}, \texttt{vStrt},
\texttt{drStrt}, and \texttt{dvStrt} are described in
section~\ref{SubSubSect:BLE:InstAttr}.
These arguments are passed directly to \texttt{BeamLineElement}.
The \texttt{SectorDipole} arguments are translated into instance
attributes as described in section~\ref{SubSubSect:SDpl:InstAttr} and
defined in table~\ref{Tab:SDpl:Attributes}.

The orientation of the RLPC coordinate axes with respect to those of
the laboratory frame changes from the start of sector dipole to its
end.
Referring to figure~\ref{fig:Dipole}, the vector, $\bm{v}_{\rm ES}$,
that translates the origin of the RLPC coordinate system at the start
of the sector dipole to the origin of the RLPC coordinate system at
its end is given by:
\begin{equation}
  \bm{v}_{\rm ES} = 2 \rho_0 \sin\left( \frac{\phi}{2} \right)
                     \begin{pmatrix}
                       \sin\left( \frac{\phi}{2} \right) \\
                       0                                 \\
                       \cos\left( \frac{\phi}{2} \right)
                     \end{pmatrix} \,;
\end{equation}
where $\rho_0$ is the radius of the circular locus of the trajectory
of the reference particle.
If the rotation matrix taking the RPLC axes at the start of the sector
dipole to the laboratory coordinate axes is
$\underline{\underline{R}}_{\rm S}$, then the vector,
$\bm{v}^{\rm lab}_{\rm ES}$, that translates from the start 
of the sector dipole to its end in laboratory coordinates is given by:
\begin{equation}
  \bm{v}^{\rm lab}_{\rm ES} = \underline{\underline{R}}_{\rm S} \bm{v}_{\rm ES} \,.
\end{equation}
The rotation matrix that transforms from the RPLC system at the end
of the sector dipole to the laboratory coordinate system,
$\underline{\underline{R}}_{\rm E}$ is given by:
\begin{equation}
  \underline{\underline{R}}_{\rm E} =
      \underline{\underline{R}}_{\rm S} \underline{\underline{R}} \, ;
\end{equation}
where:
\begin{equation}
  \underline{\underline{R}} = 
        \begin{pmatrix}
          \cos \phi & 0 & -\sin \phi \\
          0         & 1 &  0         \\
          \sin \phi & 0 &  \cos \phi 
        \end{pmatrix} \, .
\end{equation}

\paragraph{Instance attributes and access methods} ~\newline
\label{SubSubSect:SDpl:InstAttr}
\noindent
The instance attributes are defined in
table~\ref{Tab:SDpl:Attributes}. 
The attributes are accessed and set using the methods defined in
table~\ref{Tab:SDpl:Methods}.
\begin{table}[h]
  \caption{
    Definition of attributes of instances of
    the \texttt{SectorDipole(BeamLineElement)} derived class.
    All attributes are required in the call to instantiate the
    element.
  }
  \label{Tab:SDpl:Attributes}
  \begin{center}
    \begin{tabular}{|l|c|c|p{10cm}|}
      \hline
      \textbf{Attribute}   & \textbf{Type} & \textbf{Unit} & \textbf{Comment}                      \\
      \hline
      \texttt{Angle}   & float & rad & Angle through which sector dipole bends positive reference
                                 particle.                                                         \\
      \texttt{B}       & float & T   & Magnetic field.                                             \\
      \hline
    \end{tabular}
  \end{center}
\end{table}
\begin{table}[h]
  \caption{
    Definition of access methods for the \texttt{SectorDipole} derived
    class. 
  }
  \label{Tab:SDpl:Methods}
  \begin{center}
    \begin{tabular}{|c|c|p{7cm}|}
      \hline
      \textbf{Set method} & \textbf{Get method}  & \textbf{Comment}       \\
      \hline
      \texttt{setAngle(Angle)}     & \texttt{getAngle()}  & Set/get bending angle.         \\
      \texttt{setB(B)}             & \texttt{getB()}      & Set/get dipole magnetic field. \\
      \texttt{setLength()}         & \texttt{getLength()} & Set/get length of reference
                                                            particle trajectory through
                                                            sector dipole (arc length).    \\
      \texttt{setTransferMatrix()} &                      & Set transfer matrix.           \\
      \hline
    \end{tabular}
  \end{center}
\end{table}

\paragraph{Processing methods} ~\newline
\noindent
The \texttt{SectorDipole} derived class has no processing methods.

\paragraph{I/o methods} ~\newline
\noindent
Methods to read and write instance attributes to the files defined
using the \texttt{BeamIO} package (see section \ref{Sect:BeamIO} are
provided.
The calls are:
\begin{equation}
  \texttt{readElement(dataFILE)} \quad \quad {\rm and }
      \quad \quad \texttt{writeElement(dataFILE)} \,; \nonumber
\end{equation}
where \texttt{dataFILE} is the a file instance managed by \texttt{BeamIO}.

\paragraph{Utilities} ~\newline
\noindent
The \texttt{SectorDipole} derived class has no utilities.

\FloatBarrier

\subsubsection{Derived class: \texttt{Solenoid(BeamLineElement)}}

\paragraph{Instantiation} ~\newline
\noindent
The call to instantiate the \texttt{Solenoid} derived class is:
\begin{center}
  \texttt{Solenoid(Name, rStrt, vStrt, drStrt, dvStrt,
          Length, Strength, kSol)} 
\end{center}
Parent class arguments \texttt{Name}, \texttt{rStrt}, \texttt{vStrt},
\texttt{drStrt}, and \texttt{dvStrt} are described in
section~\ref{SubSubSect:BLE:InstAttr}.
These arguments are passed directly to \texttt{BeamLineElement}.
The \texttt{Solenoid} arguments are translated into instance
attributes as described in section~\ref{SubSubSect:Slnd:InstAttr} and
defined in table~\ref{Tab:Slnd:Attributes}.
The solenoid \texttt{Length} is required together with either the
magnetic field strength, \texttt{Strength} or the solenoid $k$
parameter, \texttt{kSol} (equation~\ref{Eq:Effectiveks}).  

\paragraph{Instance attributes and access methods} ~\newline
\label{SubSubSect:Slnd:InstAttr}
\noindent
The instance attributes are defined in
table~\ref{Tab:Slnd:Attributes}. 
The attributes are accessed and set using the methods defined in
table~\ref{Tab:Slnd:Methods}.
\begin{table}[h]
  \caption{
    Definition of attributes of instances of
    the \texttt{Solenoid(BeamLineElement)} derived class.
    All attributes are required in the call to instantiate the
    element.
  }
  \label{Tab:Slnd:Attributes}
  \begin{center}
    \begin{tabular}{|l|c|c|p{10cm}|}
      \hline
      \textbf{Attribute}   & \textbf{Type} & \textbf{Unit} & \textbf{Comment}                     \\
      \hline
      \texttt{Length}   & float & m     & Effective length of solenoid.                           \\
      \texttt{Strength} & float & T/m   & Magnetic field gradient; required if kSol is not given. \\
      \texttt{kSol}     & float & m$^{-2}$ & GaborLens $k$ paramter required if \texttt{Strength}
                                          not given.                                              \\
      \hline
    \end{tabular}
  \end{center}
\end{table}
\begin{table}[h]
  \caption{
    Definition of access methods for the \texttt{Solenoid} derived
    class. 
  }
  \label{Tab:Slnd:Methods}
  \begin{center}
    \begin{tabular}{|c|c|p{7cm}|}
      \hline
      \textbf{Set method} & \textbf{Get method}  & \textbf{Comment}                                  \\
      \hline
      \texttt{setLength(Length)}   & \texttt{getLength()}   & Set/get length.                        \\
      \texttt{setStrength(B)}      & \texttt{getStrength()} & Set/get strength (solenoid magnetic
                                                              field).                                \\
      \texttt{setKSol(Length)}     & \texttt{getKFQ()}      & Set/get kSol, solenoid $k$ parameter.  \\
      \texttt{setTransferMatrix()} &                        & Set transfer matrix.                   \\
      \hline
    \end{tabular}
  \end{center}
\end{table}

\paragraph{Processing methods} ~\newline
\noindent
The \texttt{Solenoid} processing method is defined in
table~\ref{Tab:Slnd:Methods}.
\begin{table}[h]
  \caption{
    Utilities provided by the \texttt{Solenoid} derived
    class. 
  }
  \label{Tab:Slnd:Methods}
  \begin{center}
    \begin{tabular}{|l|c|c|p{7cm}|}
      \hline
      \textbf{Method} & \textbf{Argument(s)} & \textbf{Return} & \textbf{Comment}                      \\
      \hline
      \texttt{calckSol()}     &  & float & Calculates \texttt{kSol} if strength is specified.          \\
      \texttt{calcStrength()} &  & float & Calculates \texttt{Strength} if \texttt{kSol} is specified. \\
      \hline
    \end{tabular}
  \end{center}
\end{table}

\paragraph{I/o methods} ~\newline
\noindent
Methods to read and write instance attributes to the files defined
using the \texttt{BeamIO} package (see section \ref{Sect:BeamIO} are
provided.
The calls are:
\begin{equation}
  \texttt{readElement(dataFILE)} \quad \quad {\rm and }
      \quad \quad \texttt{writeElement(dataFILE)} \,; \nonumber
\end{equation}
where \texttt{dataFILE} is the a file instance managed by \texttt{BeamIO}.

\paragraph{Utilities} ~\newline
\noindent
The \texttt{Solenoid} derived class has no utilities.

\FloatBarrier

\subsubsection{Derived class: \texttt{GaborLens(BeamLineElement)}}

\paragraph{Instantiation} ~\newline
\noindent
The call to instantiate the \texttt{GaborLens} derived class is:
\begin{center}
  \texttt{GaborLens(Name, rStrt, vStrt, drStrt, dvStrt,
          Bz, VA, RA, Rp, Length, kSol)}
\end{center}
Parent class arguments \texttt{Name}, \texttt{rStrt}, \texttt{vStrt},
\texttt{drStrt}, and \texttt{dvStrt} are described in
section~\ref{SubSubSect:BLE:InstAttr}.
These arguments are passed directly to \texttt{BeamLineElement}.
The \texttt{GaborLens} arguments are translated into instance
attributes as described in section~\ref{SubSubSect:GbrLns:InstAttr} and
defined in table~\ref{Tab:GbrLns:Attributes}. 
The Gabor lens \texttt{Length} is required together with either the
paramters \texttt{Bz}, \texttt{VA}, \texttt{RA}, \texttt{Rp}
correspond, respectively, to the paramters $B_z$, $V_A$, $V_A$ and
$R_p$ defined in section~\ref{SubSect:TrnsFrMtrx:GL}, or $kSol$, the
solenoid strength parameter of the equaivalent solenoid (see
section~\ref{SubSect:TrnsFrMtrx:GL}).
The effective electon number density inside the trap is calculated
using either \texttt{Bz}, \texttt{VA}, \texttt{RA} and \texttt{Rp}
or \texttt{kSol}.

\paragraph{Instance attributes and access methods} ~\newline
\label{SubSubSect:GbrLns:InstAttr}
\noindent
The instance attributes are defined in
table~\ref{Tab:GbrLns:Attributes}. 
The attributes are accessed and set using the methods defined in
table~\ref{Tab:GbrLns:Methods}.
\begin{table}[h]
  \caption{
    Definition of attributes of instances of
    the \texttt{GaborLens(BeamLineElement)} derived class.
    All attributes are required in the call to instantiate the
    element.
  }
  \label{Tab:GbrLns:Attributes}
  \begin{center}
    \begin{tabular}{|l|c|c|p{10cm}|}
      \hline
      \textbf{Attribute}   & \textbf{Type} & \textbf{Unit} & \textbf{Comment}                    \\
      \hline
      \texttt{Bz}   & float & T     & Effective length of Gabor lens.                            \\
      \texttt{VA}   & float & V     & Effective length of Gabor lens.                            \\
      \texttt{RA}   & float & m     & Effective length of Gabor lens.                            \\
      \texttt{RP}   & float & m     & Effective length of Gabor lens.                            \\
      \texttt{Length}   & float & m     & Effective length of Gabor lens.                         \\
      \texttt{Strength} & float & T/m   & Magnetic field gradient; required if kSol is not given. \\
      \texttt{kSol}     & float & m$^{-2}$ & $k$ parameter of the solenoid with the equivalent
                                            focusing strength.                                    \\
      \hline
    \end{tabular}
  \end{center}
\end{table}
\begin{table}[h]
  \caption{
    Definition of access methods for the \texttt{GaborLens} derived
    class. 
  }
  \label{Tab:GbrLns:Methods}
  \begin{center}
    \begin{tabular}{|c|c|p{7cm}|}
      \hline
      \textbf{Set method} & \textbf{Get method}  & \textbf{Comment}                                                 \\
      \hline
      \texttt{setBz(Bz)}         & \texttt{getBz()}     & Set/get magnetic field of the Penning-Malmberg trap.      \\
      \texttt{setVA(VA)}         & \texttt{getVA()}     & Set/get anode voltage of the Penning-Malmberg trap.       \\
      \texttt{setRA(RA)}         & \texttt{getRA()}     & Set/get radius of the anode of the Penning-Malmberg trap. \\
      \texttt{setRP(RP)}         & \texttt{getRP()}     & Set/get magnetic effective radiius of the plasma confined 
                                                          within the Penning-Malmberg trap.                         \\
      \texttt{setLength(Length)} & \texttt{getLength()} & Set/get effective length of the lens.                     \\

      \texttt{setStrength(Strength)} & \texttt{getStrength()} & Set/get k-parameter of the solenoid with the
                                                                equivalent focal length.                            \\
      \texttt{setElectronDenisty()} & \texttt{getElectronDenisty()} & Set/get electron density.                     \\
      \texttt{setTransferMatrix()} &                    & Set transfer matrix.                                      \\
      \hline
    \end{tabular}
  \end{center}
\end{table}

\paragraph{Processing methods} ~\newline
\noindent
The \texttt{GaborLens} dericed class has no processing methods.

\paragraph{I/o methods} ~\newline
\noindent
Methods to read and write instance attributes to the files defined
using the \texttt{BeamIO} package (see section \ref{Sect:BeamIO} are
provided.
The calls are:
\begin{equation}
  \texttt{readElement(dataFILE)} \quad \quad {\rm and }
      \quad \quad \texttt{writeElement(dataFILE)} \,; \nonumber
\end{equation}
where \texttt{dataFILE} is the a file instance managed by \texttt{BeamIO}.

\paragraph{Utilities} ~\newline
\noindent
The \texttt{GaborLens} derived class has no utilities.

\FloatBarrier

\subsubsection{Derived class: \texttt{CylindricalRFCavity(BeamLineElement)}}

\paragraph{Instantiation} ~\newline
\noindent
The call to instantiate the \texttt{CylindricalRFCavity} derived class is:
\begin{center}
  \texttt{CylindricalRFCavity(Name, rStrt, vStrt, drStrt, dvStrt,
          Gradient, Frequency, Phase)}
\end{center}
Parent class arguments \texttt{Name}, \texttt{rStrt}, \texttt{vStrt},
\texttt{drStrt}, and \texttt{dvStrt} are described in
section~\ref{SubSubSect:BLE:InstAttr}.
These arguments are passed directly to \texttt{BeamLineElement}.
The \texttt{CylindricalRFCavity} arguments are translated into instance
attributes as described in section~\ref{SubSubSect:CylndclRFCvty:InstAttr} and
defined in table~\ref{Tab:CylndclRFCvty:Attributes}. 

\paragraph{Instance attributes and access methods} ~\newline
\label{SubSubSect:CylndclRFCvty:InstAttr}
\noindent
The instance attributes are defined in
table~\ref{Tab:CylndclRFCvty:Attributes}. 
The attributes are accessed and set using the methods defined in
table~\ref{Tab:CylndclRFCvty:Methods}.
\begin{table}[h]
  \caption{
    Definition of attributes of instances of
    the \texttt{CylindricalRFCavity(BeamLineElement)} derived class.
    All attributes are required in the call to instantiate the
    element.
  }
  \label{Tab:CylndclRFCvty:Attributes}
  \begin{center}
    \begin{tabular}{|l|c|c|p{10cm}|}
      \hline
      \textbf{Attribute}   & \textbf{Type} & \textbf{Unit} & \textbf{Comment}                    \\
      \hline
      \texttt{Gradient}          & float & MV/m  & Peak electric field gradient on axis.         \\
      \texttt{Frequency}         & float & MHz   & Resonant frequency.                           \\
      \texttt{Phase}             & float & rad   & Phase cavity at time reference particle crosses
                                                   centre of cavity, ``linac convention''.       \\
      \texttt{TransitTimeFactor} & float &       & Transit time factor (equation~\ref{}).        \\
      \texttt{V0}                & float & MV    & Peak voltage.                                 \\
      \texttt{alpha}             & float &       & $\alpha$ paramter defined in equation~\ref{}. \\
      \texttt{wperp}             & float &       & $\omega_\perp$ paramter defined in equation~\ref{}. \\
      \texttt{cperp}             & float &       & $c_\perp$ paramter defined in equation~\ref{}. \\
      \texttt{sperp}             & float &       & $s_\perp$ paramter defined in equation~\ref{}. \\
      \texttt{wprll}             & float &       & $\omega_{||}$ paramter defined in equation~\ref{}. \\
      \texttt{cprll}             & float &       & $c_{||}$ paramter defined in equation~\ref{}. \\
      \texttt{sprll}             & float &       & $s_{||}$ paramter defined in equation~\ref{}. \\
      \hline
    \end{tabular}
  \end{center}
\end{table}
\begin{table}[h]
  \caption{
    Definition of access methods for the \texttt{CylindricalRFCavity} derived
    class. 
  }
  \label{Tab:CylndclRFCvty:Methods}
  \begin{center}
    \begin{tabular}{|c|c|p{7cm}|}
      \hline
      \textbf{Set method} & \textbf{Get method}  & \textbf{Comment}                                                 \\
      \hline
      \texttt{setGradient(Gradient)}        & \texttt{getGradient()}         & Set/get peak electric field gradient.      \\
      \texttt{setFrequency(Frequency)}      & \texttt{getFrequency()}        & Set/get frequency.       \\
      \texttt{setAngularFrequency(AngFreq)} & \texttt{getAngularFrequency()} & Set/get angular frequency. \\
      \texttt{setPhase(Phase)}              & \texttt{getPhase()}            & Set/get phase. \\ 
      \texttt{setWaveNumber(WaveNumber)}    & \texttt{getWaveNumber()}       & Set/get wavenumber.                     \\
      \texttt{setLength(Length)}            & \texttt{getLength()}           & Set/get Length. \\
      \texttt{setRadius(Radius)}            & \texttt{getRadius()}           & Set/get Radius. \\
      \texttt{setTransitTimeFactor(TransitTimeFactor)}            & \texttt{getTransitTimeFactor()}           & Set/get TransitTimeFactor. \\
      \texttt{setV0(V0)}                    & \texttt{getV0()}               & Set/get peak voltage.                     \\
      \texttt{setalpha(alpha)}              & \texttt{getalpha()}            & Set/get alpha.                     \\
      \texttt{setwperp(wperp)}              & \texttt{getwperp()}             & Set/get wperp.                     \\
      \texttt{setcperp(cperp)}              & \texttt{getcperp()}             & Set/get cperp.                     \\
      \texttt{setsperp(sperp)}              & \texttt{getsperp()}             & Set/get sperp.                     \\
      \texttt{setwprll(wprll)}              & \texttt{getwprll()}             & Set/get wprll.                     \\
      \texttt{setcprll(cprll)}              & \texttt{getcprll()}             & Set/get cprll.                     \\
      \texttt{setsprll(sprll)}              & \texttt{getsprll()}             & Set/get sprll.                     \\
      \texttt{setmrf(mrf)}                  & \texttt{getmrf()}              & Set/get mrf.                     \\
      \texttt{setTransferMatrix()}          &                                & Set transfer matrix.               \\
      \hline
    \end{tabular}
  \end{center}
\end{table}

\paragraph{Processing methods} ~\newline
\noindent
The \texttt{CylindricalRFCavity} dericed class has no processing methods.

\paragraph{I/o methods} ~\newline
\noindent
Methods to read and write instance attributes to the files defined
using the \texttt{BeamIO} package (see section \ref{Sect:BeamIO} are
provided.
The calls are:
\begin{equation}
  \texttt{readElement(dataFILE)} \quad \quad {\rm and }
      \quad \quad \texttt{writeElement(dataFILE)} \,; \nonumber
\end{equation}
where \texttt{dataFILE} is the a file instance managed by \texttt{BeamIO}.

\paragraph{Utilities} ~\newline
\noindent
The \texttt{CylindricalRFCavity} derived class has no utilities.

\FloatBarrier

\subsection{\texttt{BeamLine}}

\texttt{BeamLine} is a singleton class that sets up the beam line
geometry and provides methods to track particles through the beam line
using the transfer matarices defined in section~\ref{SubSubSect:BLE}.
The beam-line geometry is provided in the form of a ``\texttt{csv}''
file read using \texttt{pandas}.
The format of the ``\texttt{csv}'' file is defined in
section~\ref{SubSubSect:??}.
Alternatively, if a data file written using the \texttt{BeamIO}
package is being read, the beam-line geometry is read from the top of
the data file.

\subsubsection{Instantiation} ~\newline
\noindent
The call to instantiate the \texttt{BeamLineElenent} class is:
\begin{center}
  \texttt{BeamLine(BeamLineSpecificationCSVfile, readDataFile)}
\end{center}
\texttt{BeamLineSpecificationCSVfile} is a the full path of the CSV
file containing the beam-line specificiation.
\texttt{readDataFile} is a boolean flag.
If \texttt{readDataFile} is set to \texttt{True}, then the \texttt{BeamLine}
instance will be created and the beam-line geometry will be read from
the header of the BeamIO data file.
If \texttt{readDataFile} is not set or is set to \texttt{False}, the
beam-line geometry will be read
from \texttt{BeamLineSpecificationCSVfile}.

\subsubsection{Instance attributes and access methods} ~\newline
\label{SubSubSect:BLE:InstAttr}
\noindent
The instance attributes are presented in table~\ref{Tab:BL:Attributes}
and the access methods are summarised in table~\ref{Tab:BL:AccessMethods}.
\begin{table}[h]
  \caption{
    Definition of attributes of instances of
    the \texttt{BeamLine} class.
  }
  \label{Tab:BL:Attributes}
  \begin{center}
    \begin{tabular}{|l|c|c|p{10cm}|}
      \hline
      \textbf{Attribute} & \textbf{Type} & \textbf{Comment}                                                                  \\
      \hline
      \texttt{BeamLineSpecificationCSVfile}$^*$ & path      & Full path to beam-line specification \texttt{csv} file.        \\
      \texttt{BeamLineParamPandasInstance}      & dataframe & Pandas data frame containing beam-line specification.          \\
      \texttt{Element}                          & list      & List of instances of \texttt{BeamLineElement} class containing
                                                              pointers to the elements that make up the beam line.           \\
      \hline
    \end{tabular}
  \end{center}
\end{table}
\begin{table}[h]
  \caption{
    Definition of access methods for the \texttt{BeamLine}
    class. 
  }
  \label{Tab:BL:AccessMethods}
  \begin{center}
    \begin{tabular}{|c|c|p{7cm}|}
      \hline
      \textbf{Set method} & \textbf{Get method}  & \textbf{Comment}                                                              \\
      \hline
      \texttt{setSrcTrcSpc(SrcTrcSpc)} & \texttt{setSrcTrcSpc()} & Set trace space at source; SrcTrcSpc presented as \texttt{(1,6)}
                                                                   \texttt{np.ndarray}.                                          \\
                                         & \texttt{getinsance()} & Get instance of \texttt{BeamLine} class.                        \\
                       & \texttt{BeamLineSpecificationCSVfile()} & Get beam line specification csv file.                           \\
                             & \texttt{(getBeamLineParamPandas)} & Get pandas dat from containing beam-line specification.         \\
                                         & \texttt{getElement()} & get list of \texttt{BeamLineElement} instances.                 \\
      \hline
    \end{tabular}
  \end{center}
\end{table}

\subsubsection{Processing methods} ~\newline
\noindent
Table~\ref{Tab:BL:ProcMethods} presents the processing methods provided
in the \texttt{BeamLine} class.
\begin{table}[h]
  \caption{
    Processing methods provided by the \texttt{BeamLine}
    class. 
  }
  \label{Tab:BL:ProcMethods}
  \begin{center}
    \begin{tabular}{|l|c|c|p{7cm}|}
      \hline
      \textbf{Method} & \textbf{Argument(s)} & \textbf{Return} & \textbf{Comment}                                            \\
      \hline
      \texttt{addBeamLine()}   &  & \texttt{Success} & Loops through pandas data frame and manages parsing and instanciation
                                                       of the beam line elements defined in the specification \texttt{csv} file.
                                                       Returns \texttt{Success} (bool) which is \texttt{True} if the beamline
                                                       has been set up OK and is \texttt{False} otherwise.                     \\
      \texttt{addFacility()}   &  &  & Manages the extraction of the facility paramters from the pandas data frame and the
                                       creation of the single instance of \texttt{Facility(BeamLineElement)}.                  \\
      \texttt{addSource()}     &  &  & Manages the extraction of the source paramters from the pandas data frame and the
                                       creation of the single instance of \texttt{Source(BeamLineElement)}.                  \\
      \texttt{parseFacility()} &  & \texttt{Name}, \texttt{K0}, \texttt{VCMVr} & Parses pandas data frane to extract
                                                                                 facility parameters.
                                                                                 Returns the facility \texttt{Name} (str), the
                                                                                 kinetic energy of the reference particle,
                                                                                 \texttt{K0} (float) in MeV, and the vacuum
                                                                                 chamber mother volume radius, \texttt{VCMVr}
                                                                                 (float) in m.                                 \\
      \texttt{parseSource()}   &  & \texttt{Name}, \texttt{Mode}, \texttt{Param} & Parses pandas data frane to extract
                                                                                 source parameters.
                                                                                 Returns \texttt{Name} (str), \texttt{Mode} (int)
                                                                                 \texttt{Param} (list) containing the parameters
                                                                                 required to instanciate source \texttt{Mode}. \\
      \texttt{addBeamLineElement(iBLE)} &  &  & Adds \texttt{BeamLineElement} instance \texttt{iBLE} to the list of instances of
                                                     \texttt{BeamLineElement} that make up the beam line.                      \\
      \texttt{checkConsistency()} &  & \texttt{Consistent} & Checks the consistency of the beam line representation in memory
                                                             with that requested in the specification \texttt{csv} file.
                                                             Returns \texttt{Consistent} (bool) which is \texttt{True} if the
                                                             beamline is consistent is \texttt{False} otherwise.                     \\
      \texttt{trackBeamn(NEvts, particleFILE)} &  &  & Generates \texttt{NEvts} (int) particles and tracks them through the
                                                       beam line. \\
      \hline
    \end{tabular}
  \end{center}
\end{table}

\subsubsection{I/o methods} ~\newline
\noindent
Methods to read and write instance attributes to the files defined
using the \texttt{BeamIO} package (see section \ref{Sect:BeamIO} are
provided.
The calls are:
\begin{equation}
  \texttt{readElement(dataFILE)} \quad \quad {\rm and }
      \quad \quad \texttt{writeElement(dataFILE)} \,; \nonumber
\end{equation}
where \texttt{dataFILE} is the a file instance managed by \texttt{BeamIO}.
In addition the \texttt{csv} file containing the specification of the
beam line is read in using the method:
\begin{equation}
  \texttt{csv2pandas(csvFILE)} \,; \nonumber
\end{equation}
where \texttt{csvFILE} (path) is the full path to the \texttt{csv}
file.

\subsubsection{Utilities} ~\newline
\noindent
The \texttt{BEamLine} class has no utilities.

\FloatBarrier

\subsection{Particle and reference particle}
\label{SubSect:BLE:ParticleReferenceParticle}

The \texttt{Particle} class provides methods to transport particles
through the beam line.
The trace and phase space is recorded at the start and end of each
element.
The \texttt{ReferenceParticle} derived class is a singleton and
records the trajectory of the reference particle.

\subsubsection{\texttt{Particle}} ~\newline
\label{SubSubSect:BLE:Particle}

\paragraph{Instantiation}  ~\newline
\noindent
The call to instantiate the \texttt{Particle} class is:
\begin{center}
  \texttt{Particle(Species)}
\end{center}
\texttt{Species}, the type of particle to be propagated, is a string
containing the particle name.
At present valid particle \texttt{species} are proton, pion, and
muon.

\paragraph{Instance attributes and access methods} ~\newline
\label{Para:BLE:InstAttr}
\noindent
The instance attributes are presented in table~\ref{Tab:P:Attributes}
and the access methods are summarised in table~\ref{Tab:P:AccessMethods}.
\begin{table}[h]
  \caption{
    Definition of attributes of instances of the \texttt{Particle} class.
  }
  \label{Tab:P:Attributes}
  \begin{center}
    \begin{tabular}{|l|c|c|p{10cm}|}
      \hline
      \textbf{Attribute} & \textbf{Type} & \textbf{Comment}                                                                  \\
      \hline
      \texttt{Species}       & str   & Particle species; proton, muon or pion.                                                     \\
      \texttt{Location}      & list  & List of strings containing the unique \texttt{Name} of the \texttt{BeamLineElement} at
                                       the particle position is reported.                                                          \\
      \texttt{s}             & list  & List of floats recording $s$ coordinate at which particle position is reported.             \\
      \texttt{TraceSpace}    & list  & List of \texttt{np.ndarray} containing 6D trace space of particle at \texttt{s}.            \\
      \texttt{PhaseSpace}    & list  & List of \texttt{np.ndarray} containing 6D phase space (RPLC) of particle at \texttt{s}.     \\
      \texttt{LabPhaseSpace} & list  & List of \texttt{np.ndarray} containing 6D phase space (Lab) of particle at \texttt{s}.      \\
      \hline
    \end{tabular}
  \end{center}
\end{table}
\begin{table}[h]
  \caption{
    Definition of access methods for the \texttt{Particle}
    class. 
  }
  \label{Tab:P:AccessMethods}
  \begin{center}
    \begin{tabular}{|c|c|p{7cm}|}
      \hline
      \textbf{Set method} & \textbf{Get method}  & \textbf{Comment}                                                      \\
      \hline
      \texttt{setSpecies}        & \texttt{getSpecies}        & Set/get particle species.                                \\
      \texttt{setLocation}       & \texttt{getLocation}       & Set/get list of locations location.                      \\
      \texttt{sets}              & \texttt{gets}              & Set/get list of $s$ coordinates.                         \\
      \texttt{setTraceSpace}     & \texttt{getTraceSpace}     & Set/get list of trace-space vectors.                     \\
      \texttt{setRPLCPhaseSpace} & \texttt{setRPLCPhaseSpace} & Set/get list of phase-space vectirs in RPLC coordinates. \\
      \texttt{setLabPhaseSpace}  & \texttt{getLabPhaseSpace}  & Set/get list of phase-space vectirs in Lab coordinates.  \\
      \hline
    \end{tabular}
  \end{center}
\end{table}

\paragraph{Processing methods} ~\newline % Here!
\noindent
Table~\ref{Tab:P:ProcMethods} presents the processing methods provided
in the \texttt{Particle} class.
\begin{table}[h]
  \caption{
    Processing methods provided by the \texttt{Particle}
    class. 
  }
  \label{Tab:P:ProcMethods}
  \begin{center}
    \begin{tabular}{|l|c|c|p{7cm}|}
      \hline
      \textbf{Method} & \textbf{Argument(s)} & \textbf{Return} & \textbf{Comment}                                            \\
      \hline
      \texttt{initialiseSums()} &  &  & Initialises sums used to calculate covariance matrix.                                \\
      \texttt{incrementSums(iPrtcl)}  & \texttt{Particle} instance &  & Increment sums used to calculate covariance matrix.  \\
      \texttt{calcCovarianceMatrix()} &  &  & Calculate covariance matrix.                                                   \\
      \texttt{evaluateBeam()} &  &  & Work through locations and calculate paramters from covariance matrix.                 \\
      \hline
    \end{tabular}
  \end{center}
\end{table}

\paragraph{I/o methods} ~\newline
\noindent
The \texttt{Particle} class has no i/o methods.

\paragraph{Utilities} ~\newline
\noindent
The utilities provided by the \texttt{Particle} class are listed in
table~\ref{Tab:P:UtilMethods}.
\begin{table}[h]
  \caption{
    Utilities provided by the \texttt{Particle}
    class. 
  }
  \label{Tab:P:UtilMethods}
  \begin{center}
    \begin{tabular}{|l|c|c|p{7cm}|}
      \hline
      \textbf{Method} & \textbf{Argument(s)} & \textbf{Return} & \textbf{Comment}                                          \\
      \hline
      \texttt{cleanBeams()}       &  &  & Delete (using \texttt{del}) instances of beam class.          \\
      \texttt{printProgression()} &  &  & Prints the paramters at each location.                                           \\
      \texttt{getHeader()}        &  &  & Returns header for pandas data frame used to store summary.                      \\
      \texttt{getLines()}         &  &  & Returns lines to be used to create summary pandas data frame.                    \\
      \texttt{createReport()}     &  &  & Creates \texttt{csv} file containing summary of beam progression.                \\
      \hline
    \end{tabular}
  \end{center}
\end{table}

\FloatBarrier
