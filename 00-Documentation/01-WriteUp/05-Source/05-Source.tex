\graphicspath{ {03-Source/Figures/} }

\section{Source}

Para 1:
\begin{itemize}
  \item A variety of options for generating the particle distribution
    at source are included in the package;
  \item The principle (default) option is the model of the TNSA
    mechanism presented in \cite{PhysRevLett.90.185002}.
\end{itemize}

Para 2:
\begin{itemize}
  \item The energy spectrum of the protons per unit energy and unit
    solid angle is given by:
    \begin{equation}
      \frac{dN}{dE} = \frac{n_{i0}c_st}{(2EE_0)^{\frac{1}{2}}}
                         \exp\left[
                   -\left( \frac{2E}{E_0} \right)^{\frac{1}{2}} \right]\, ;
      \label{Eq:Spct:0}
    \end{equation}
    where $n_{i0}$ is the ion number density at $t=0$, the instant the
    laser pulse strikes the target; $c_s$ is given by:
    \begin{equation}
      c_s = \left[ \frac{Z k_{\rm B} T_e}{m_i} \right]^{\frac{1}{2}} \, ;
    \end{equation}
    where $Z$ is the ion charge number, $k_{\rm B}$ is the Boltzmann
    constant; $T_e$ is the electron temperature; $m_i$ is the ion mass;
    $t$ is the instant in time at which the spectrum is evaluated; and
    \begin{equation}
      E_0 = \left[ \frac{n_{e0} k_{\rm B} T_e}{\epsilon_0} \right]^{\frac{1}{2}}
    \end{equation}
\end{itemize}

Para 3:
\begin{itemize}
  \item To generate the proton spectrum at the source, a practical
    approach is taken.
    The leading behaviour of equation \ref{Eq:Spct:0} is taken to be:
    \begin{equation}
      \frac{dN}{dE} = \Gamma
        \frac{\exp\left[ -E^{\frac{1}{2}} \right]}{E^{\frac{1}{2}}}\, ;
      \label{Eq:Spct:1}
    \end{equation}
  \item Normalisation of equation~\ref{Eq:Spct:1} between a minimum
    energy ($E_{\rm min}$) and a maximum energy ($E_{\rm max}$) is
    used to determine the constant $\Gamma$.
    Both $E_{\rm min}$ and $E_{\rm max}$ are user defined and set at
    the initialisation stage.
\end{itemize}
