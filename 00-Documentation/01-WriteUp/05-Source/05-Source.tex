\graphicspath{ {05-Source/Figures/} }

\section{Source}

A variety of options for the generation of the particle distribution
at source are included in the package (see
section \ref{SourceOptions}.
The principal, and the default, option ios the target-normal sheath
acceleration (TNSA) model presented in~\cite{10.1038/nphys199}.
This model is summarised in this section.

\subsection{Energy distribution}

The typical energy spectrum produced in target-normal sheath
acceleration falls rapidly with energy before dropping rapidly to zero
above a maximumm ``cut off'' energe $\varepsilon_{\rm max}$.
The model energy spectrum of the TNSA model presented
in~\cite{10.1038/nphys199} is given by:
\begin{equation}
  \frac{dN}{d\varepsilon} = \frac{n_{e0} c_{s} t_{laser} S_{sheath}}
                                 {\sqrt{2\varepsilon T_{e}}}
                                 exp(- \sqrt{\frac{2\varepsilon}{T_{e}}})\,;
  \label{Eq:Spct:0}
\end{equation}
where $N$ is the number of protons or ions produced per unit solid
angle, $\varepsilon$ is the ion energy in Joules, $n_{e0}$ and $T_e$
are the hot electron density and temperature respectively, $c_s$ is
the ion-acoustic velocity, $t_{\rm laser}$ is the duration of the
laser pulse, and $S_{\rm sheath}$ is the effective area over which the
TNSA mechanism takes place.
The variables and the units in which they are expressed are presented
in table~\ref{table:EnergySpectrumParameters}.
\begin{table} 
  \caption{
    Parameters present in the TNSA analytical
    equation \ref{Eq:Spct:0}.
  }
  \label{table:EnergySpectrumParameters}
  \begin{center}
    \begin{tabular}{c c c c}
      \hline
      \textbf{Parameter} & \textbf{Definition} & \textbf{Value} & \textbf{Unit} \\ [1ex] 
      \hline \hline
      $N$ & Ion number & - & - \\ 
      $\varepsilon$ & Ion energy & - & J \\  
      $n_{e0}$ & Hot electron density & $\frac{N_{E}}{c t_{laser} S_{sheath}}$ & $pp/m^3$ \\ 
      $N_{e}$ & Accelerated electron number & $\frac{f E_{laser}}{T_e}$ & - \\ 
      $E_{laser}$ & Laser energy & $70$ & J \\  
      $f$ & Energy conversion efficiency & $1.2 \times 10^{-15} I^{0.75}$, max=0.5  & - \\   % valid only for I < 3.1x10^19 W/cm2
      $I$ & Laser intensity & $4 \times 10^{20}$ & $W/cm^{2}$ \\ 
      $T_{e}$ & Hot electron temperature & $m_{e} c^{2} [\sqrt{1 + \frac{I \lambda^{2}}{1.37 \times 10^{18}} -1} ]$ & J \\ 
      $m_{e}$ & Electron mass & $9.11 \times 10^{-31}$ & Kg \\ 
      $c$ & Speed of light & $3 \times 10^{8}$ & m/s \\ 
      $\lambda$ & Laser wavelength & 0.8 & $\mu$m \\  
      $t_{laser}$ & Laser pulse duration & $28 \times 10^{-15}$  & s \\  
      $B$ & Radius of electron bunch & $B=r_{0} + d tan(\theta)$ & $m$ \\ 
      $S_{sheath}$ & Electron acceleration area & $\pi B^{2}$ & $m^{2}$ \\ 
      $r_{0}$ & Laser spot radius & $\sqrt{\frac{P_{laser}}{I \pi}}$, I in $W/m^{2}$ & m \\  
      $d$ & Target thickness & $400-600 \times 10^{-9}$ & m \\  
      $\theta$ & Electron half angle divergence & 0.436 & rad \\  
      $P_{laser}$ & Laser power & $2.5 \times 10^{15}$, $P_{laser}=\frac{E_{laser}}{t_{laser}}$ & W \\  
      $c_{s}$ & Ion-acoustic velocity & $(\frac{Z k_{B} T_{e}}{m_{i}})^{\frac{1}{2}}$ & m/s \\  
      $Z$ & Ion charge number & 1 & - \\  
      $k_{B}$ & Boltzmann constant & $1.380649 \times 10^{-23}$ & $m^{2} kg s^{-2} K^{-1}$ \\  
      $m_{i}$ & Proton mass & $1.67 \times 10^{-27}$ & Kg \\ 
      $P_{R}$ & Relativistic power unit & $\frac{m_{e} c^{2}}{r_{e}} = 8.71 \times 10^{9}$ & W \\  
      $r_{e}$ & Electron radius & $2.82 \times 10^{-15}$ & m \\  
      $\varepsilon_{i,\infty}$ & Maximum ion energy & $2 Z m_{e} c^{2} \sqrt{\frac{f P_{laser}}{P_{R}}}$ & MeV \\  
      $t_{0}$ & Ballistic time & $\frac{B}{v(\infty)}$ & s \\  
      $v(\infty)$ & Ballistic velocity & $\sqrt{\frac{2 \varepsilon_{i,\infty}}{m_{i}}}$ & m/s \\  
      \hline
    \end{tabular}
  \end{center}
\end{table}

Equation~\ref{Eq:Spct:0} is based on time-limited fluid models which
are unable to predict the cut-off energy, $\varepsilon_{\rm max}$,
accurately.
The cut-off energy is taken to be that given by the model described
in~\cite{10.1103/PhysRevLett.97.045005} where the time over which the
laser pulse creates the conditions necessary for acceleration.
The energy cut-off is given by:
\begin{equation}
  \varepsilon_{max} = X^{2} \varepsilon_{i,\infty} \, ;
  \label{eq:Eq:Spct:2}
\end{equation}
where $X$ is obtained by solving:
\begin{equation}
  \frac{t_{laser}}{t_{0}} = X \left( 1 + \frac{1}{2}
                           \frac{1}{1 - X^{2}} \right) +
                           \frac{1}{4} \ln \left( \frac{1+X}{1-X} \right) \, ;
  \label{eq:Eq:Spct:1}
\end{equation}
where $t_0$ is the time over which the ion acceleration may be treated
as ballistic and $\varepsilon_{i,\infty}$ is given in
table~\ref{table:EnergySpectrumParameters}.


  To generate the energy spectrum, a practical approach is taken:
  \begin{itemize}
    \item Equation~\ref{Eq:Spct:0} is normalized.
    \item A random number is generated between 0 and 1 for the y-axis.
    \item If the number falls below the equation it gets accepted, otherwise the loop is repeated.
    \item If accepted, the corresponding x-value (energy) is calculated and returned.
  \end{itemize}



\subsection{Angular Distribution}

The angular distribution of the particles at the source has been approximated with a Gaussian distribution \cite{10.1038/s41598-019-41705-0}. 
The following approach has been used to generate it:

\begin{itemize}
  \item The maximum divergence angle of the protons has been calculated based on the particle's energy.
  \item A linear equation has been used that generates a maximum divergence angle of 20 degrees for the low energies, down to 5 degrees for the cut-off.
  \item A gaussian distribution with a FWHM equal to the divergence angle has been used for each particle. 
  \item A divergence angle is returned that falls within that distribution.
\end{itemize}

A beam diameter of 10 microns has been used.
