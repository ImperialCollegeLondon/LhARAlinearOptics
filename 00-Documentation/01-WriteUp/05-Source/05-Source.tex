\graphicspath{ {05-Source/Figures/} }

\section{Source}

A variety of options for the generation of the particle distribution
at source are included in the package (see
section \ref{SourceOptions}.
The principal, and the default, option is the target-normal sheath
acceleration (TNSA) model presented in~\cite{10.1038/nphys199}.
This model is summarised in this section.

\subsection{Energy distribution}

The typical kinetic energy spectrum produced in target-normal sheath
acceleration falls rapidly with kinetic energy before dropping rapidly
to zero above a maximum ``cut off'' energy $\varepsilon_{\rm max}$.
The model kinetic-energy spectrum of the TNSA model presented
in~\cite{10.1038/nphys199} is given by:
\begin{equation}
  \frac{dN}{d\varepsilon} = \frac{n_{e0} c_{s} t_{laser} S_{sheath}}
                                 {\sqrt{2\varepsilon T_{e}}}
                                 exp(- \sqrt{\frac{2\varepsilon}{T_{e}}})\,;
  \label{Eq:Spct:0}
\end{equation}
where $N$ is the number of protons or ions produced per unit solid
angle, $\varepsilon$ is the ion kinetic energy in Joules, $n_{e0}$ and
$T_e$ are the hot electron density and temperature respectively, $c_s$
is the ion-acoustic velocity, $t_{\rm laser}$ is the duration of the
laser pulse, and $S_{\rm sheath}$ is the effective area over which the
TNSA mechanism takes place.
The variables and the units in which they are expressed are presented
in table~\ref{table:EnergySpectrumParameters}.
\begin{table} 
  \caption{
    Parameters present in the TNSA analytical
    equation \ref{Eq:Spct:0}.
  }
  \label{table:EnergySpectrumParameters}
  \begin{center}
    \begin{tabular}{c c c c}
      \hline
      \textbf{Parameter} & \textbf{Definition} & \textbf{Value} & \textbf{Unit} \\ [1ex] 
      \hline \hline
      $N$ & Ion number & - & - \\ 
      $\varepsilon$ & Ion kinetic energy & - & J \\  
      $n_{e0}$ & Hot electron density & $\frac{N_{E}}{c t_{laser} S_{sheath}}$ & $pp/m^3$ \\ 
      $N_{e}$ & Accelerated electron number & $\frac{f E_{laser}}{T_e}$ & - \\ 
      $E_{laser}$ & Laser energy & $70$ & J \\  
      $f$ & Energy conversion efficiency & $1.2 \times 10^{-15} I^{0.75}$, max=0.5  & - \\   % valid only for I < 3.1x10^19 W/cm2
      $I$ & Laser intensity & $4 \times 10^{20}$ & $W/cm^{2}$ \\ 
      $T_{e}$ & Hot electron temperature & $m_{e} c^{2} [\sqrt{1 + \frac{I \lambda^{2}}{1.37 \times 10^{18}} -1} ]$ & J \\ 
      $m_{e}$ & Electron mass & $9.11 \times 10^{-31}$ & Kg \\ 
      $c$ & Speed of light & $3 \times 10^{8}$ & m/s \\ 
      $\lambda$ & Laser wavelength & 0.8 & $\mu$m \\  
      $t_{laser}$ & Laser pulse duration & $28 \times 10^{-15}$  & s \\  
      $B$ & Radius of electron bunch & $B=r_{0} + d tan(\theta)$ & $m$ \\ 
      $S_{sheath}$ & Electron acceleration area & $\pi B^{2}$ & $m^{2}$ \\ 
      $r_{0}$ & Laser spot radius & $\sqrt{\frac{P_{laser}}{I \pi}}$, I in $W/m^{2}$ & m \\  
      $d$ & Target thickness & $400-600 \times 10^{-9}$ & m \\  
      $\theta$ & Electron half angle divergence & 0.436 & rad \\  
      $P_{laser}$ & Laser power & $2.5 \times 10^{15}$, $P_{laser}=\frac{E_{laser}}{t_{laser}}$ & W \\  
      $c_{s}$ & Ion-acoustic velocity & $(\frac{Z k_{B} T_{e}}{m_{i}})^{\frac{1}{2}}$ & m/s \\  
      $Z$ & Ion charge number & 1 & - \\  
      $k_{B}$ & Boltzmann constant & $1.380649 \times 10^{-23}$ & $m^{2} kg s^{-2} K^{-1}$ \\  
      $m_{i}$ & Proton mass & $1.67 \times 10^{-27}$ & Kg \\ 
      $P_{R}$ & Relativistic power unit & $\frac{m_{e} c^{2}}{r_{e}} = 8.71 \times 10^{9}$ & W \\  
      $r_{e}$ & Electron radius & $2.82 \times 10^{-15}$ & m \\  
      $\varepsilon_{i,\infty}$ & Maximum ion kinetic energy & $2 Z m_{e} c^{2} \sqrt{\frac{f P_{laser}}{P_{R}}}$ & MeV \\  
      $t_{0}$ & Ballistic time & $\frac{B}{v(\infty)}$ & s \\  
      $v(\infty)$ & Ballistic velocity & $\sqrt{\frac{2 \varepsilon_{i,\infty}}{m_{i}}}$ & m/s \\  
      \hline
    \end{tabular}
  \end{center}
\end{table}

Equation~\ref{Eq:Spct:0} is based on time-limited fluid models which
are unable to predict the cut-off kinetic energy, $\varepsilon_{\rm max}$,
accurately.
The cut-off kinetic energy is taken to be that given by the model
described in~\cite{10.1103/PhysRevLett.97.045005} where the time over
which the laser pulse creates the conditions necessary for
acceleration. 
The kinetic energy cut-off is given by:
\begin{equation}
  \varepsilon_{max} = X^{2} \varepsilon_{i,\infty} \, ;
  \label{eq:Eq:Spct:2}
\end{equation}
where $X$ is obtained by solving:
\begin{equation}
  \frac{t_{laser}}{t_{0}} = X \left( 1 + \frac{1}{2}
                           \frac{1}{1 - X^{2}} \right) +
                           \frac{1}{4} \ln \left( \frac{1+X}{1-X} \right) \, ;
  \label{eq:Eq:Spct:1}
\end{equation}
where $t_0$ is the time over which the ion acceleration may be treated
as ballistic and $\varepsilon_{i,\infty}$ is given in
table~\ref{table:EnergySpectrumParameters}.

To generate the kinetic energy spectrum, the probability density
function, $g(\varepsilon)$, is defined such that the probability,
$\delta \cal{P}$, of a particle being generated in the interval
$\varepsilon \rightarrow \varepsilon + \delta \varepsilon$ is given
by:
\begin{equation}
   \delta \cal{P} = g(\varepsilon) \delta \varepsilon \, .
\end{equation}
$g(\varepsilon)$ can be written in terms of the differential spectrum
given in equation~\ref{Eq:Spct:0} through the introduction of a
normalisation constant $\cal{N}$:
\begin{equation}
  g(\varepsilon) = \frac{1}{\cal{N}} \frac{dN}{d\varepsilon} \, .
\end{equation}
The cumulative distribution funtion, $G(\varepsilon)$, is given by:
\begin{equation}
  G(\varepsilon) = \int_0^{\varepsilon_{\rm max}} g(\varepsilon)
                                               d\varepsilon \,;
\end{equation}
where the normalisation constant, $\cal{N}$, is set so that
$G(\varepsilon_{\rm max}) = 1$.
Carrying out the intergration yields:
\begin{equation}
  G(\varepsilon) = \frac{2}{\cal{N}}
                   \frac{n_{e0} c_{s} t_{laser} S_{sheath}} {\sqrt{2T_{e}}}
                   \sqrt{\frac{T_{e}}{2}}
                   \left[
                     1 - \left(exp(- \sqrt{\frac{2\varepsilon}{T_{e}}}\right)
                   \right] \, .
\end{equation}

The kinetic energy spectrum may now be obtained by choosing a value
for $\varepsilon$ using a probability distribution uniform over the
range $0 < \varepsilon < \varepsilon_{max}$.   
A second random number, $\eta$, uniformly distributed on the range
$0 < \eta < 1$ is then chosen.
If $\eta$ is greater than $G(\varepsilon)$, a new value for
$\varepsilon$ is chosen and the process repeats.
Alternatively, if $\eta$ is less than $G(\varepsilon)$, the value of
$\varepsilon$ is accepted. 

\subsection{Angular Distribution}

The angular distribution of the flux of protons and ions produced by
the TNSA mechanism may be described as a cone centred on the normal to
the foil surface.
The opening angle of the cone decreases as the ion energy considered
increases.

The angular distribution of the particles at the laser-driven source
has been approximated using a Gaussian
distribution~\cite{10.1038/s41598-019-41705-0}.  
At low kinetic energy ($\varepsilon = 0$), the sigma
($\sigma(\varepsilon)$) of the gaussian distribution is taken to be
$20^\circ$.
$\sigma(\varepsilon)$ is assumed to decrease linearly with energy such
that
\begin{equation}
  \sigma(\varepsilon) = 20^\circ - (\varepsilon_{max}
                           - \varepsilon)\times15 \, ;
  \label{Eq:sigVeps}
\end{equation}
i.e. $\sigma(\varepsilon)$ decreases from $20^\circ$ at
$\varepsilon=0$ to $5^\circ$ at $\varepsilon_{max}$.
The polar angle, $\theta$, is then chosen from the gaussian
distribution with sigma given by equation~\ref{Eq:sigVeps}.

Finally, the azimuthal angle, $\phi$, is chose from a distribution
uniform over the range $0 < \phi < 2\pi$.

{\color{red} Maria, not sure what this means: A beam diameter of 10
microns has been used.}
